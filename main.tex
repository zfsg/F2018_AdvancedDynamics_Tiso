
\documentclass[10pt,a4paper]{scrartcl}

\usepackage[english]{babel}

\input{../Headerfiles/Packages}
\input{../Headerfiles/Titles}
\input{../Headerfiles/Commands}
\graphicspath{{Pictures/}}
\parindent 0pt

\title{Advanced Dynamics}
\author{GianAndrea Müller}

\newtheorem{define}{Definition}

\begin{document}
\begin{multicols*}{4}
\maketitle
\tableofcontents
\end{multicols*}

\begin{multicols*}{2}
\section{Dynamics}
\subsection{Coordinate frames}

\subsubsection{Inertial coordinate frame}

A frame of reference (coordinate system) that is not rotating and not accelerated.

\subsubsection{Cartesian coordinate frame}

\myspic{0.45}{Pictures/CartesianCoordinates}
\begin{equation*}
\vec{r}=\begin{pmatrix}x(t)\\y(t)\\z(t)\end{pmatrix}
\enspace
{\vec{v}=\begin{pmatrix}\dot{x}(t)\\\dot{y}(t)\\\dot{z}(t)\end{pmatrix}}
\enspace
{\vec{a}=\begin{pmatrix}\ddot{x}(t)\\\ddot{y}(t)\\\ddot{z}(t)\end{pmatrix}}
\end{equation*}


\subsubsection{Cylindrical coordinate frame}

\myspic{0.45}{Pictures/CylindricalCoordinates}

\begin{equation*}
{\vec{r}=\begin{pmatrix} \rho(t) \\ \theta(t) \\ z(t) \\ \end{pmatrix}} 
\enspace
{\vec{v}=\begin{pmatrix} \dot{\rho}(t)  \\ \rho(t)\dot{\theta}(t) \\ \dot{z}(t) \\ \end{pmatrix}} 
\enspace
{\vec{a}=\begin{pmatrix} \ddot{\rho}(t)-\rho(t)\dot{\theta}(t)^2 \\ 2 \dot{\rho}(t)\dot{\theta}(t)+\rho(t)\ddot{\theta}(t) \\ \ddot{z}(t) \\ \end{pmatrix}}
\end{equation*}

\subsubsection{Spherical coordinate frame}

\begin{center}
\def\svgwidth{\linewidth}
\input{Pictures/SphericalCoordinates.pdf_tex}
\end{center}

\begin{equation*}
{\vec{r}=\begin{pmatrix} \rho(t) \\ \theta (t) \\ \psi (t) \\ \end{pmatrix}}   
\enspace
{\vec{v}=\begin{pmatrix} \dot{\rho} \\ \rho \dot{\theta} cos(\psi) \\ \rho \dot{\psi} \\ \end{pmatrix}}   
\enspace
{\vec{a}=\begin{pmatrix}  \ddot{\rho}-\rho \dot{\theta}^2 \cos^2 \psi - \rho \dot{\psi}^2  \\ 2 \dot{\rho} \dot{\theta} \cos(\psi) + \rho \ddot{\theta} \cos(\psi) - 2 \rho \dot{\theta} \dot{\psi} \sin(\psi)  \\ 2 \dot{\rho} \dot{\psi} + \rho \dot{\theta} \sin( \psi) \cos( \psi) + \rho \ddot{\psi} \\ \end{pmatrix}}
\end{equation*}

\subsection{Constraints}

A constraint is a scalar relation that limits the possible displacement of particles.

\begin{enumerate}[label=(\alph*)]
\item \textbf{Rigid link}

\[|\vec{r}_{AB}| = \sqrt{(x_A-x_B)^2+(y_A-y_B)^2+(z_A-z_B)^2} = const\]
\item \textbf{Rotating tube}

\[\begin{pmatrix}
x_A\\y_A\\z_A
\end{pmatrix}=\rho_A(t)\vec{e}(t)
\]

where $\vec{e}(t)$ denotes the direction of the rotating tube.
\item \textbf{Planar motion}

\[z_A = const\]
\item \textbf{Rolling without slipping}
\[\dot{x}=R\dot{\phi}\]
\end{enumerate}

\textbf{Holonomic constraint:}

A holonomic constraint is a constraint that only depends on the current
position of the particles involved and the current time. A non-holonomic constraint
also depends on time-derivatives of the motion.

\subsection{Generalized coordinates}

A set of variables used to uniquely describe the configuration of a system
are called generalized coordinates. This can comprise, e.g., distances or angles.

\subsection{Degree of freedom}

The degree of freedom (DOF) of a system is defined as the number of
independent generalized coordinates necessary to describe the motion of the system
completely and uniquely.

\begin{equation*}
\text{DOF} = dim\cdot n -k
\end{equation*}

\definitiontable{
$dim$&dimension of the space\\
$n$&number of particles in the system\\
$k$&number of holonomic constraints applied to the system
}

\begin{itemize}
\item There are always different possibilities of interpreting the DOF of a multi body system. Be sure that you define the constraints such that each body you consider is only moving in the exact way it is supposed to.
\end{itemize}

\subsection{Givens rotation}

Givens rotation matrices define a rotation about one of the three coordinate directions about some angle $\alpha,\beta$, or $\gamma$ referring to a rotation about the x-axis, y-axis or z-axis respectively. 

\begin{equation*}
\begin{aligned}
\tvec{R}_x&=\begin{pmatrix}1&0&0\\0&\cos\alpha&-\sin\alpha\\0&\sin\alpha&\cos\alpha\end{pmatrix}\\
\tvec{R}_y&=\begin{pmatrix}\cos\beta&0&-\sin\beta\\0&1&1\\\sin\beta&0&\cos\beta\end{pmatrix}\\
\tvec{R}_z&=\begin{pmatrix}\cos\gamma&-\sin\gamma&0\\\sin\gamma&\cos\gamma&0\\0&0&1\end{pmatrix}
\end{aligned}
\end{equation*}

\subsection{Taylor series}

The Taylor series of a function $f(x)$, which is infinitely differentiable at some number $a$ is given by,

\begin{equation*}
f(a)+\frac{f'(a)}{1!}(x-a)+\frac{f''(a)}{2!}(x-a)^2+\frac{f'''(a)}{3!}(x-a)^2+\cdots,
\end{equation*}

or equivalently,

\begin{equation*}
\sum\limits_{n=0}^\infty \frac{f^{(n)}(a)}{n!}(x-a)^n.
\end{equation*}

\subsubsection{Small angle approximation}

Based on a Taylor series of the second order some trigonometrical functions can be approximated well for small angles.

\begin{center}
\begin{tabular}{l@{$\;$}l}
$\sin\theta$&$\approx\theta$\\
$\cos\theta$&$\approx1-\frac{\theta^2}{2}$\\
$\tan\theta$&$\approx\theta$
\end{tabular}
\end{center}

\section{Single-particle dynamics}

\subsection{Linear momentum}

\begin{equation*}
\vec{P}:=m\vec{v}
\end{equation*}

\definitiontable{
$\vec{P}$&linear momentum &in& $\si{\kilo\gram\meter\per\second}$\\
$m$&mass&in&$\si{\kilo\gram}$\\
$v$&velocity&in&$\si{\meter\per\second}$
}

\subsection{Principle of linear momentum for a single particle (LMP)}
The linear momentum principle is equivalent to Newton's second axiom. It states that a resultant forces acting on a particle causes an acceleration of the particle.

\begin{equation*}
\vecd{P} = m\vec{a} = \vec{F}
\end{equation*}

\definitiontable{
$\vecd{P}$&change of linear momentum &in& $\si{\kilo\gram\meter\per\second\squared}$\\
$m$&mass&in&$\si{\kilo\gram}$\\
$\vec{a}$&acceleration&in&$\si{\meter\per\second\squared}$\\
$\vec{F}$&resultant force&in&$\si{N}=\si{\kilo\gram\meter\per\second\squared}$
}

\begin{itemize}
\item Use for problems based on linear motion.
\item Use to find equation of motion based on the acting forces or find the acting forces based on a defined motion.
\item If no resultant force is acting on the particle, the acceleration is zero and the linear momentum of the particle is conserved.
\item The knowledge of all forces acting on a particle leads directly to its acceleration. If the initial velocity is known, the particle's velocity can be derived as a function of time. If the initial position is known additionally, the particle's position can be calculated as well.
\item The LMP is described by a vectorial equation. Therefore you can set up a scalar equation for each coordinate direction.
\item Only valid in an inertial frame!
\end{itemize}

\vspace{0.5cm}
\textbf{Remark on LMP in cylindrical coordinates:}

In case of a curvilinear motion where cylindrical coordinates are helpful, you need to express the vector $m \vec{a}$ as a function of the cylindrical coordinates $(\rho,\theta)$ and project it to the radial and azimuthal unit vectors $\vec{e}_\rho$,$\vec{e}_{\theta}$ afterwards. With this trick you get the linear momentum principle in cylindrical coordinates (forces acting on the particle are then expressed in the $\vec{e}_\rho$,$\vec{e}_{\theta}$ system).

A special case of this is what has been derived in Problem 2.2, where we treat a circular motion with a constant velocity.

\subsection{Angular momentum}

\myspic{0.4}{Pictures/AngularMomentum}

\begin{equation*}
\vec{H}_B(t):=\vec{\varrho}_B(t)\times\vec{P}(t)
\end{equation*}


\definitiontable{
$\vec{H}_B(t)$&angular momentum w.r.t. point B&in&$\si{\kilo\gram\meter\squared\per\second}$\\
$\vec{\rho}_B(t)$&vector between point $B$ and the particle m&in&$\si{\meter}$\\
$\vec{P}(t)$&linear momentum of particle m&in&$\si{\kilo\gram\meter\per\second}$
}

\subsection{Principle of angular momentum for a single particle w.r.t. $B$ (AMP)}
The angular momentum principle can be seen as Newton's second axiom for rotation. It states that the angular momentum of a particle is influenced by the torque acting on the particle. The angular momentum itself can be calculated w.r.t an arbitrary reference point $B$ which can be fixed or moving.

\begin{equation*}
\vecd{H}_B+\vec{v}_B\times\vec{P} = \vec{M}_B
\end{equation*}

\definitiontable{
$\vecd{H}_B$&change angular momentum w.r.t. $B$ &in& $\si{\kilo\gram\meter\squared\per\second\squared}$\\
$\vec{v}_B$&velocity of $B$&in&$\si{\meter\per\second}$\\
$\vec{P}$&linear momentum&in&$\si{\kilo\gram\meter\per\second}$\\
$\vec{M}_B$&resultant torque w.r.t. $B$&in&$\si{\newton\meter}=\si{\kilo\gram\meter\squared\per\second\squared}$
}
\begin{itemize}
\item Often used if a particle is in curvilinear motion.
\item If $\vec{v}_B =0$ or $\vec{v}_B || \vec{P}$ and $\vec{M}_B=0$, the angular momentum w.r.t point $B$ is conserved. 
\item If the angular momentum is conserved and the particle's initial velocity is given, it is often useful to equate the initial angular momentum with the particle's angular momentum at a later time. 
\item Choose points with $\vec{v}_B = 0$ to avoid calculating the terms of the cross product. 
\item Although the AMP is a vectorial equation, it is convenient to write it in a scalar form if a 2D problem needs to be solved. In the 2D case, the angular momentum and therefore it's derivative are always normal to the plane of projection.
\end{itemize}

\subsection{Work-energy principle for a single particle}
A force acting on a moving particle will change the particle's kinetic energy $T_i$. 

\begin{equation*}
W_{12}=T_2-T_1
\end{equation*}

\definitiontable{
$W_{12}$&work done on particle between point 1 and 2&in& $\si{\joule} = \si{\kilo\gram\meter\squared\per\second\squared}$\\
$T_i$&Kinetic energy at point i&in&$\si{\kilo\gram\meter\squared\per\second\squared}$
}
\begin{itemize}
\item The kinetic energy of a particle is given by,
\begin{equation*}
T_i = \frac{1}{2}mv_i^2.
\end{equation*}
\item Used if only the initial and end state of a particle is of interest and not the states in between.
\item Can easily be calculated if the force acting on the particle is parallel to the particle's velocity.
\item The work done by a force is zero, if the force is normal to the particle's velocity.
\item The work done on a particle by a force is given by,
	
	\begin{equation*}
	W = \int_{P_1}^{P_2}\vec{F}\cdot \vec{dr} \ =\ \int_{P_1}^{P_2}|\vec{F}|\cos(\alpha)dS.
	\end{equation*}  
\end{itemize}

\subsection{Definition of a potential force}
A force is called potential force if a potential functions exists, such that the negative gradient of this function is equal to the force.

\begin{equation*}
\vec{F}=-\frac{\partial V}{\partial \vec{r}}=-\nabla V(\vec{r})
\end{equation*}

\definitiontable{
$\vec{F}$&potential force&in& $\si{\newton}=\si{\kilo\gram\meter\per\second\squared}$\\
$V$&potential function&in&$\si{\newton\meter}=\si{\kilo\gram\meter\squared\per\second\squared}$\\
$\vec{r}$&position vector&in&$\si{\meter}$
}
If a force has a potential, the work done by this force between two points does not depend on the path between the points. In this case, the work done by the force can be calculated by the difference of the values of the potential function in this points, i.e., $W_{12} = V_1 - V_2 = V(x_1,y_1,z_1) - V(x_2,y_2,z_2)$.

\vspace{12pt}
An example for a potential force is the gravitational force. It's potential is given by,
\begin{equation*}
V(\vec{r}) = mgz,
\end{equation*} 
because, 
\begin{equation*}
-\nabla V(\vec{r}) = \begin{pmatrix}
0 \\ 0 \\ -mg \end{pmatrix}.
\end{equation*}

\subsection{Work-energy principle for a conservative system}
The work energy principle for a conservative system holds true, if all forces acting on a system are either potential (=conservative) or do no work. In this case, the work done by potential forces can be expressed as the difference of the potentials at the start and the end point of the desired motion.

\begin{equation*}
T_1+V_1 =T_2+V_2\quad\Leftrightarrow\quad \ddt[T(t)+V(t)]=0
\end{equation*}

\definitiontable{
$T_i$&Kinetic energy in point i&in&$\si{\joule}=\si{\kilo\gram\meter\squared\per\second\squared}$\\
$V_i$&Potential energy in point i&in&$\si{\joule}=\si{\kilo\gram\meter\squared\per\second\squared}$\\
$T(t)$&Kinetic energy at time t&in&$\si{\joule}=\si{\kilo\gram\meter\squared\per\second\squared}$\\
$V(t)$&Potential energy at time t&in&$\si{\joule}=\si{\kilo\gram\meter\squared\per\second\squared}$
}

\begin{itemize}
\item Always check if the system of interest is conservative
\item A conservative system has the advantage that you do not need to know the path of the particle to calculate the work done by the forces acting on it.
\item If some of the forces are potential and some not, express the work done by the potential forces as a difference of the potentials and calculate the work done by the non-conservative forces with the standard procedure, if you know the path of the particle.
\item Some common examples for potentials are given by:

\begin{center}
	\begin{tabular}{lr}
		Force & Potential \\
		\hline
		Gravitation & $mgz$ \\
		Linear spring & $-\frac{1}{2}kx^2$ \\
		Torsional spring & $-\frac{1}{2}k\theta^2$
	\end{tabular}
\end{center}
\end{itemize}

\subsection{Mass accretion equation}

\myspic{0.2}{Rocket}

The mass accretion equation describes the motion of a particle that loses or gains mass and is under the influence of some external force. The loss or gain of mass is described by a change of momentum $\dot{m}\vec{u}$ where $\vec{u}$ denotes the velocity of the gained or ejected mass.

\begin{equation*}
\dot{m}\vec{v}+m\vecd{v}-\dot{m}\vec{u}=\vec{F}\quad\Leftrightarrow\quad m\vecd{v}=\vec{F}+\dot{m}\vec{v}_{rel}
\end{equation*}

\definitiontable{
$m$&mass depending on time&in&$\si{\kilo\gram}$\\
$\dot{m}$&change of mass&in&$\si{\kilo\gram\per\second}$\\
$\vec{v}$&velocity of the particle&in&$\si{\meter\per\second}$\\
$\vec{u}$&velocity of the ejected or gained mass&in&$\si{\meter\per\second}$\\
$\vecd{v}$&acceleration&in&$\si{\meter\per\second\squared}$\\
$\vec{F}$&resultant force&in&$\si{\newton}=\si{\kilo\gram\meter\per\second\squared}$\\
$\vec{v}_{rel} = \vec{u}-\vec{v}$&relative velocity&in&$\si{\meter\per\second}$
}

An exemplary application of this principle is a rocket that fights gravity while it is loosing mass. In that case the acting force is the gravitational force, $v$ represents the velocity of the rocket, $\vec{u}$ represents the velocity of the exhaust gases. Note that the gravitational force varies as well since the mass of the rocket varies.

\subsection{General transformation to a moving, rotating y-frame}

Such a frame can be described as follows:

\sbs{0.5}{0.4}{
\begin{center}
\def\svgwidth{0.8\linewidth}
\input{Pictures/NonInertial.pdf_tex}
\end{center}
}{
\[\tvec{Q}(t)=\tvec{R}_z=\begin{pmatrix}\cos\theta&-\sin\theta&0\\\sin\theta&\cos\theta&0\\0&0&1\end{pmatrix}\]
}

\begin{equation*}
\vec{x}=\tvec{Q}(t)\vec{y}+\vec{b}(t)
\end{equation*}

\definitiontable{
$\vec{x}$&position vector in inertial x-frame&in&$\si{\meter}$\\
$\vec{y}$&position vector in non-inertial y-frame&in&$\si{\meter}$\\
$Q(t)$&rotation matrix that aligns the $\vec{y}$ axes with the $\vec{x}$ axes\\
&$\tvec{Q}^T=\tvec{Q}^{-1}$ and $\det\left(\tvec{Q}\right)=1$\\
$b(t)$&translational vector between the two origins of the coordinate systems&in&$\si{\meter}$
}

\subsection{Linear momentum principle in a planar, rotating frame}

Considering particles in non-inertial frames we have to consider the influence of non-inertial forces

\begin{equation*}
\begin{aligned}
m\vecdd{y}&=\tilde{\vec{F}}+\vec{F}_{centrifugal}+\vec{F}_{Coriolis}+\vec{F}_{Euler}\\
\vec{F}_{centrifugal}&=m\vec{\Omega}^2\vec{y}\\
\vec{F}_{Coriolis}&=-2m\vec{\Omega}\times\vecd{y}\\
\vec{F}_{Euler}&=-m\vecd{\Omega}\times\vec{y}
\end{aligned}
\end{equation*}

\definitiontable{
$m$&mass&in&$\si{\kilo\gram}$\\
$\vecd{y}$&velocity in y-frame&in&$\si{\meter\per\second}$\\
$\vecdd{y}$&acceleration in y-frame&in&$\si{\meter\per\second\squared}$\\
$\tilde{\vec{F}}$&active and constraint force w.r.t. the moving frame&in&$\si{\newton}=\si{\kilo\gram\meter\squared\per\second\squared}$\\
$\vec{\Omega}$&angular velocity of the y-frame&in&$\si{\radian\per\second}$\\
$\vecd{\Omega}$&angular acceleration of the y-frame&in&$\si{\radian\per\second\squared}$\\
}

\begin{itemize}
\item Note that the y-frame denotes the moving frame.
\end{itemize}

\section{Multi-particle dynamics}

\subsection{Center of mass for a system of particles}

The center of mass represents the location in which the resultant gravitational force of all included particles attacks. Or in other words it is the point in which the resultant moment of all gravitational forces is zero.

\begin{equation*}
\vec{r}_C=\frac{1}{M}\sum\limits_{i=1}^n m_i\vec{r}_i,\quad \text{where}\ M=\sum\limits_{i=1}^n m_i
\end{equation*}

\definitiontable{
$\vec{r}_C$&center of mass&in&$\si{\meter}$\\
$M$&total mass of the system&in&$\si{\kilo\gram}$\\
$m_i$&individual masses of particles&in&$\si{\kilo\gram}$\\
$\vec{r}_i$&locations of particles&in&$\si{\meter}$\\
$n$&number of particles
}

\subsection{Linear momentum of a system of particles}

\begin{equation*}
\vec{P}:=\sum\limits_{i=1}^n\vec{P}_i=\sum\limits_{i=1}^nm_i\vec{v}_i
\end{equation*}

\subsection{Resultant force on a system of particles}

\begin{equation*}
\vec{F}:=\sum\limits_{i=1}^n\vec{F}_i^{int}+\sum\limits_{i=1}^n\vec{F}_i^{ext}
\end{equation*}

\begin{itemize}
\item Note that the sum of the inner forces is zero by definition.
\end{itemize}

\subsection{Linear momentum principle for a system of particles}

The linear momentum principle for a system of particles is equivalent to Newton's second axiom. It states that a resultant forces acting on such a system causes an acceleration of the center of mass.

\begin{equation*}
\vecd{P}=M{\vec{a}_C}=\sum\limits_{i=1}^n \vec{F}_i^{ext}=\vec{F}^{ext}
\end{equation*}

\definitiontable{
$\vecd{P}$&change of linear momentum&in&$\si{\kilo\gram\meter\per\second\squared}$\\
$M$&total mass of the system&in&$\si{\kilo\gram}$\\
$a_C$&acceleration of the center of mass&in&$\si{\meter\per\second\squared}$\\
$\vec{F}_i^{ext}$&individual resultant forces on particles&in&$\si{\newton}=\si{\kilo\gram\meter\per\second\squared}$\\
$\vec{F}^{ext}$&total resultant force on system&in&$\si{\newton}=\si{\kilo\gram\meter\per\second\squared}$\\
$n$&number of particles&&
}

\begin{itemize}
\item If no resultant force is acting on the system, the acceleration of the center of mass is zero and the linear momentum of the system is conserved.
\item If all external forces on a system are known the acceleration of the center of mass can be determined. Vice versa if the acceleration of the center of mass is known the sum of the external forces can be calculated.
\item Since the linear momentum principle is a vectorial equation it can be split up and considered in each coordinate direction separately. This means also that it can also be conserved in one direction while it changes in the other direction.
\item Only valid in an inertial frame!
\end{itemize}

\subsection{Angular momentum principle for a system of particles w.r.t. a point $B$}

The angular momentum principle is the same as in the case of a single particle. The only difference is, that $\vec{P}$ is equal to the linear momentum of the system of particles, $\vec{P} = m \vec{v}_{C}.$

\begin{equation*}
\vecd{H}_B+\vec{v}_B\times\vec{P}=\vec{M}_B^{ext}
\end{equation*}


\definitiontable{
$\vecd{H}_B$&change of angular momentum w.r.t. $B$&in&$\si{\kilo\gram\meter\squared\per\second\squared}$\\
$\vec{v}_B$&velocity of point $B$&in&$\si{\meter\per\second}$\\
$\vec{P}$&linear momentum of the system&in&$\si{\kilo\gram\meter\per\second}$\\
$\vec{M}_B^{ext}$&external torque acting on the system&in&$\si{\newton\meter}=\si{\kilo\gram\meter\squared\per\second\squared}$
}

\begin{itemize}
\item The angular momentum is conserved if and only if:
\begin{itemize}
\item $\vec{M}_B^{ext}\equiv 0$ AND either $\vec{v}_B\equiv 0$ or $\vec{P}\equiv 0$.
\item Alternatively it is possible that: $\vec{M}_B^{ext}-\vec{v}_B\times\vec{P}\equiv 0$.
\end{itemize}
\item When considering impulsive forces it makes sense to assess the change of angular momentum over an infinitesimally short time, therefore allowing negligence of non-impulsive forces.
\end{itemize}

\subsection{Collision of two particles: coefficient of restitution}

\begin{equation*}
e=-\frac{\text{relative normal velocity after impact}}{\text{relative normal velocity before impact}}=-\frac{u_2^n-u_1^n}{v_2^n-v_1^n}
\end{equation*}

\definitiontable{
$e$&coefficient of restitution&&\\
$u_i^n$&individual normal velocities after impact&in&$\si{\meter\per\second}$\\
$v_i^n$&individual normal velocities before impact&in&$\si{\meter\per\second}$
}

\begin{itemize}
\item $e=1$ implies that we are dealing with a perfectly elastic collision.
\item $e=0$ implies that we are dealing with a perfectly inelastic collision.
\item Due to energy conservation $e$ must satisfy $0\leq e\leq 1$.
\item $e$ can be interpreted as a measure of the kinetic energy preserved during the collision. The amount of energy lost goes into deformation of the bodies.
\item The linear momentum of the individual particles is always conserved in tangential direction.
\item The linear momentum of the system is always conserved, since no forces act in tangential direction and the forces in normal direction are opposite in direction and equal in magnitude and therefore cancel out.
\end{itemize}

\subsection{Velocity relation between velocities before and after impact}

\begin{alignat}{3}
u_1^n&=\frac{m_1-em_2}{m_1+m_2}v_1^n+&\frac{(1+e)m_2}{m_1+m_2}v_2^n \\
u_2^n&=\frac{(1+e)m_1}{m_1+m_2}v_1^n+&\frac{m_2-em_1}{m_1+m_2}v_2^n
\end{alignat}

\definitiontable{
$u_i^n$&individual normal velocities after impact&in&$\si{\meter\per\second}$\\
$v_i^n$&individual normal velocities before impact&in&$\si{\meter\per\second}$\\
$m_i$&individual masses&in&$\si{\kilo\gram}$\\
$e$&coefficient of restitution&&
}

\subsection{Work-energy principle for a system of particles}

\begin{equation*}
W_{12}=T_2-T_1
\end{equation*}

\begin{equation*}
T:=\sum\limits_{i=1}^n\onha m_i|\vec{v}_i|^2
\end{equation*}

\definitiontable{
$W_{12}$&total work done by all forces on the particles between point $1$ and $2$&in&$\si{\joule}=\si{\kilo\gram\meter\squared\per\second\squared}$\\
$T_i$&total kinetic energy of the system at $t_i$&in&$\si{\joule}=\si{\kilo\gram\meter\squared\per\second\squared}$
}

\begin{itemize}
\item Always check if the system is conservative. If so you can apply conservation of mechanical energy:

\[T_1 + V_1 = T_2 + V_2\]
\item If the system is not conservative split the work done in a conservative and a non-conservative part.
\item To calculate the non-conservative part of the work keep in mind the definition of work. Only the component of the force that is parallel to the velocity of its point of attack does work!

\[W_{12}^{nc}=\int_{P_1}^{P_2}\vec{F}\cdot\vec{dr}=\int_{P_1}^{P_2}|\vec{F}|\cos(\alpha)dS\]
\item Check if the system is a rigid system of particles! If yes the work done by the internal forces is zero. $\rightarrow$ \ref{sec:RigidSystem}, \ref{sec:WERigidSystem}
\end{itemize}


\subsubsection{Rigid system of particles}
\label{sec:RigidSystem}

\begin{equation*}
\ddt\left|\vec{r}_i-\vec{r}_j\right|^2 \equiv 0\ \forall\ i,j=1,\ldots,n
\end{equation*}

\definitiontable{
$\vec{i}$&location of particle i&in&$\si{\meter}$\\
$n$&number of particles
}

\subsection{Work-energy principle for a rigid system of particles}
\label{sec:WERigidSystem}

The difference compared to the work-energy principle for a general system of particles is that in the case of a rigid system of particles the work done by the internal forces equals zero.

\begin{equation*}
W_{12}^{ext}=T_2-T_1
\end{equation*}

\definitiontable{
$W_{12}^{ext}$&total work done by external forces on the particles between point $1$ and $2$&in&$\si{\joule}=\si{\kilo\gram\meter\per\second\squared}$\\
$T_i$&total kinetic energy of the system at $t_i$&in&$\si{\joule}=\si{\kilo\gram\meter\per\second\squared}$
}

\section{Rigid body dynamics}

\subsubsection{Connection vector}

\begin{equation*}
\vec{r}_{AB} = \vec{r}_B-\vec{r}_A
\end{equation*}

\subsection{Velocity transfer formula for a rigid body}

\begin{equation*}
\vec{v}_B=\vec{v}_A+\vec{\omega}\times\vec{r}_{AB}
\end{equation*}

\definitiontable{
$\vec{v}_B$&velocity in point $B$&in&$\si{\meter\per\second}$\\
$\vec{v}_A$&velocity in point $A$&in&$\si{\meter\per\second}$\\
$\vec{\omega}$&rotational velocity of $\mathcal{B}$&in&$\si{\rad\per\second}$\\
$\vec{r}_{AB}$&connection vector between $A$ and $B$&in&$\si{\meter}$
}

\begin{itemize}
\item Remember that angular velocities can be added up.
\item It is not always necessary to do vector calculations. Clever usage of the SDPG (Theorem of projected velocities) and the instantaneous center of rotation can be a solution too.
\end{itemize}

\subsubsection{Instantaneous center of rotation}

The instantaneous center of rotation $O$ of a planar rigid body $\mathcal{B}$ is defined as the point $O\in\mathcal{B}'$ for which

\begin{equation*}
\vec{v}_O=0
\end{equation*}

Hence the extended rigid body $\mathcal{B}'$ includes the original body $\mathcal{B}$ and the instantaneous center of rotation $O$.

\begin{itemize}
\item If the instantaneous center of rotation and the angular velocity of a body is known the velocity of any point that is part of the body can be easily calculated.
\end{itemize}

\subsubsection{Instantaneous axis of rotation}

As seen in Mechanics I, in any moving rigid body there can be found an axis whose points only have translational velocity. This axis is called the instantaneous axis of rotation.

\begin{figure}[h]
\mypic{InstantaneousAxisOfRotation}
\caption{Quelle: Ingenieurmechanik 1 - Grundlagen und Statik - 2. Auflage, p51}
\end{figure}

\subsection{Center of mass of a rigid body $\mathcal{B}$}

\begin{equation*}
\vec{r}_C=\frac{1}{M}\int_\mathcal{B}\vec{r}dm=\frac{1}{M}\int_\mathcal{B}\vec{r}\rho dV
\end{equation*}

\definitiontable{
$\vec{r}_C$&location of the center of mass&in&$\si{\meter}$\\
$M$&total mass of $\mathcal{B}$&in&$\si{\kilo\gram}$\\
$\rho$&density of $\mathcal{B}$&in&$\si{\kilo\gram\per\meter\cubed}$
}

\begin{itemize}
\item Normally this integral can be split up into a simple sum.
\end{itemize}

\subsection{Linear momentum principle for a rigid body $\mathcal{B}$}

\begin{equation*}
\vecd{P}=M\vec{a}_C=\vec{F}^{ext}
\end{equation*}

\definitiontable{
$\vecd{P}$&change of linear momentum of $\mathcal{B}$&in&$\si{\kilo\gram\meter\per\second\squared}$\\
$M$&total mass of $\mathcal{B}$&in&$\si{\kilo\gram}$\\
$\vec{a}_C$&acceleration of the center of mass&in&$\si{\meter\per\second\squared}$\\
$\vec{F}^{ext}$&resultant external force on the body&in&$\si{\newton}=\si{\kilo\gram\meter\per\second\squared}$
}

\begin{itemize}
\item If no resultant force is acting on the body, the acceleration of the center of mass is zero and the linear momentum of the system is conserved.
\item If all external forces on a system are known the acceleration of the center of mass can be determined. Vice versa if the acceleration of the center of mass is known the sum of the external forces can be calculated.
\item Since the linear momentum principle is a vectorial equation it can be split up and considered in each direction separately. This means also that it can be conserved in one direction while it changes in the other direction.
\item Only valid in an inertial frame!
\end{itemize}

\subsection{The angular momentum of a rigid body $\mathcal{B}$ w.r.t. a fixed point $B$, belonging to the body, or w.r.t. its center of mass $B\equiv C$}

\begin{equation*}
\vec{H}_B=\int_\mathcal{B}\vec{\varrho}_B\times(\vec{\omega}\times\vec{\varrho}_B)dm=\tvec{I}_B\vec{\omega}
\end{equation*}

\definitiontable{
$\vec{H}_B$&angular momentum of $\mathcal{B}$ w.r.t. $B$&in&$\si{\kilo\gram\meter\squared\per\second}$\\
$\vec{\varrho}_B$&distance from location of dm to point $B$&in&$\si{\meter}$\\
$\vec{\omega}$&rotational velocity of $\mathcal{B}$&in&$\si{\rad\per\second}$\\
$\tvec{I}_B$&moment of inertia of $\mathcal{B}$&in&$\si{\kilo\gram\meter\squared}$
}

\begin{itemize}
\item Remember that angular momenta w.r.t. the same point can be added up.
\item Do not forget to use the angular momentum transfer formula whenever necessary!
\end{itemize}

\subsection{Mass moments of inertia of a rigid body $\mathcal{B}$}

$\tvec{I}_B$ is the \textbf{moment of intertia tensor} of the body $\mathcal{B}$ w.r.t. the point B.

\begin{equation*}
\tvec{I}_B:=\begin{pmatrix}I_{xx}&I_{xy}&I_{xz}\\I_{xy}&I_{yy}&I_{yz}\\I_{xz}&I_{yz}&I_{zz}\end{pmatrix}
\end{equation*}

where

\begin{center}
\large
\renewcommand{\arraystretch}{2}
\begin{tabular}{l@{$\qquad$}l@{$\qquad$}l}
$I_{xx}=\int\limits_\mathcal{B}(y^2+z^2)\ dm$&
$I_{xy}=-\int\limits_\mathcal{B}xy\ dm$&
$I_{xz}=-\int\limits_\mathcal{B}xz\ dm$\\
$I_{yy}=\int\limits_\mathcal{B}(x^2+z^2)\ dm$&
$I_{yz}=-\int\limits_\mathcal{B}yz\ dm$&
$I_{zz}=\int\limits_\mathcal{B}(x^2+y^2)\ dm$
\end{tabular}
\end{center}

\begin{itemize}
\item The mass moments of inertia are always given at the exam, either as an equation or directly as a value.
\item Remember that moments of inertia formulated w.r.t. to the same point can simply be added up.
\item Remember to apply Steiner's theorem wherever necessary.
\end{itemize}

\subsection{Angular momentum principle for a rigid body $\mathcal{B}$ with respect to an arbitrary point $B$}

\begin{equation*}
\vecd{H}_B+\vec{v}_B\times\vec{P}=\vec{M}_B^{ext}
\end{equation*}

\definitiontable{
$\vecd{H}_B$&change of angular momentum w.r.t. $B$&in&$\si{\kilo\gram\meter\squared\per\second\squared}$\\
$\vec{v}_B$&velocity of point $B$&in&$\si{\meter\per\second}$\\
$\vec{P}$&linear momentum of $\mathcal{B}$&in&$\si{\kilo\gram\meter\per\second}$\\
$\vec{M}_B^{ext}$&external torque acting on $\mathcal{B}$&in&$\si{\newton\meter}=\si{\kilo\gram\meter\squared\per\second\squared}$
}

\begin{itemize}
\item $\vec{H}_B = \tvec{I}_B\cdot\vec{\omega}$ \textbf{if and only if Point B is fixed or coincides with the center of mass!}
\item The angular momentum is conserved if and only if:
\begin{itemize}
\item $\vec{M}_B^{ext}\equiv 0$ AND either $\vec{v}_B\equiv 0$ or $\vec{P}\equiv 0$.
\item Alternatively it is possible that: $\vec{M}_B^{ext}-\vec{v}_B\times\vec{P}\equiv 0$.
\end{itemize}
\item When considering impulsive forces it makes sense to assess the change of angular momentum over an infinitesimally short time, therefore allowing negligence of non-impulsive forces.
\item When applying the angular momentum principle to a rolling cylinder it is possible to choose the contact point with the surface as a point of respect, since that way you get rid of the external moment of the contact force and $\vec{v_B}\times\vec{P}$, because in that case $\vec{v}_B\parallel\vec{v}_C$. However if the cylinder still slides consider the next tip:
\item When applying the angular momentum principle in a point that is not the center of mass or that is not fixed you must use the angular momentum transfer formula!
\end{itemize}

\subsection{Angular momentum transfer formula for a rigid body $\mathcal{B}$ between two arbitrary points $A$ and $B$}

\begin{equation*}
\vec{H}_B=\vec{H}_A+\vec{P}\times\vec{r}_{AB}
\end{equation*}

\definitiontable{
$\vec{H}_B$&angular momentum w.r.t. point $B$&in&$\si{\kilo\gram\meter\squared\per\second}$\\
$\vec{H}_A$&angular momentum w.r.t. point $A$&in&$\si{\kilo\gram\meter\squared\per\second}$\\
$\vec{P}$&linear momentum of $\mathcal{B}$&in&$\si{\kilo\gram\meter\per\second}$\\
$\vec{r}_{AB}$&connection vector between $A$ and $B$
}

\subsection{Parallel axis theorem (Steiner's theorem)}

\begin{equation*}
\tvec{I}_O=\tvec{I}_C+M\left(|\vec{r}_{CO}|^2\tvec{\mathbb{I}}-\vec{r}_{CO}\vec{r}_{CO}^T\right)=\tvec{I}_C+M\begin{pmatrix}b^2+c^2&-ab&-ac\\-ab&a^2+c^2&-bc\\-ac&-bc&a^2+b^2\end{pmatrix}
\end{equation*}

\definitiontable{
$\tvec{I}_O$&moment of inertia tensor of $\mathcal{B}$ w.r.t. some point $O$&in&$\si{\kilo\gram\meter\squared}$\\
$\tvec{I}_C$&moment of inertia tensor of $\mathcal{B}$ w.r.t. the center of mass of $\mathcal{B}$&in&$\si{\kilo\gram\meter\squared}$\\
$M$&total mass of $\mathcal{B}$&in&$\si{\kilo\gram}$\\
$\vec{r}_{CO}$&connection vector between $C$ and $O$&in&$\si{\meter}$\\
$\tvec{\mathbb{I}}$&unity matrix in $\mathbb{R}^3$
}

\subsection{Parallel axis theorem (Steiner's theorem) for planar bodies}

\begin{equation*}
I_A=I_C+M|\vec{r}_{CA}|^2
\end{equation*}

\definitiontable{
$I_A$&moment of inertia of $\mathcal{B}$ w.r.t. point $A$&in&$\si{\kilo\gram\meter\squared}$\\
$I_C$&moment of inertia of $\mathcal{B}$ w.r.t. point $C$&in&$\si{\kilo\gram\meter\squared}$\\
$M$&total mass of $\mathcal{B}$&in&$\si{\kilo\gram}$\\
$\vec{r}_{CA}$&connection vector between C and A&in&$\si{\meter}$
}

\subsection{Kinetic energy $T$ of a rigid body $\mathcal{B}$}

\begin{equation*}
T=\onha M|\vec{v}_C|^2+\onha \vec{\omega}^T\tvec{I}_C\vec{\omega}
\end{equation*}

\definitiontable{
$T$&kinetic energy of $\mathcal{B}$&in&$\si{\joule}=\si{\kilo\gram\meter\squared\per\second\squared}$\\
$M$&total mass of $\mathcal{B}$&in&$\si{\kilo\gram}$\\
$\vec{v}_C$&velocity of the center of mass of $\mathcal{B}$&in&$\si{\meter\per\second}$\\
$\vec{\omega}$&rotational velocity of $\mathcal{B}$&in&$\si{\rad\per\second}$\\
$\tvec{I}_C$&moment of inertia tensor of $\mathcal{B}$ w.r.t. its center of mass&in&$\si{\kilo\gram\meter\squared}$
}

\begin{itemize}
\item For a fixed point $B$:
\begin{equation*}
T = \frac{1}{2}\vec{\omega}\tvec{I}_B\vec{\omega}
\end{equation*}
\end{itemize}

\subsection{Work-energy principle for a rigid body}

\begin{equation*}
W_{12}^{ext}=T_2-T_1
\end{equation*}

\definitiontable{
$W_{12}^{ext}$&total work done by all forces on $\mathcal{B}$&in&$\si{\joule}=\si{\kilo\gram\meter\squared\per\second\squared}$\\
$T_i$&total kinetic energy of the system at $t_i$&in&$\si{\joule}=\si{\kilo\gram\meter\squared\per\second\squared}$
}

\subsection{Differentiation of quantities in moving frames}

\begin{equation*}
\vecd{u}=\mathring{\vec{u}}+\vec{\Omega}\times\vec{u}
\end{equation*}

\definitiontable{
$\vecd{u}$&acceleration of $\mathcal{B}$ in moving frame&in&$\si{\meter\per\second\squared}$\\
$\mathring{\vec{u}}$&velocity of $\mathcal{B}$ in moving frame&in&$\si{\meter\per\second}$\\
$\vec{\Omega}$&rotational velocity of the moving frame&in&$\si{\radian\per\second}$\\
$\vec{u}$&velocity of $\mathcal{B}$&in&$\si{\meter\per\second}$
}

\begin{itemize}
\item Differentiation in moving frame makes sense since aligning the chosen coordinate system with the principle axes of the body in question simplifies calculations. The vector-matrix multiplications are much simpler if you have the principal moments of inertia instead of a general inertia tensor.
\item The rotational velocity of the moving frame can coincide with the rotational velocity of the body but in general does not have to.
\end{itemize}

\subsubsection{Principal moments of inertia}

Consider a basis $\begin{bmatrix}\vec{u}_1&\vec{u}_2&\vec{u}_3\end{bmatrix}$ in which the moment of inertia becomes diagonal.

\begin{equation*}
\tvec{I}_B=\begin{pmatrix}
I_1&0&0\\0&I_2&0\\0&0&I_3
\end{pmatrix}
\end{equation*}

$\begin{bmatrix}\vec{u}_1&\vec{u}_2&\vec{u}_3\end{bmatrix}$ are then called the principal axes of $\mathcal{B}$ and $I_i$ are the principal moments of inertia


\subsection{Euler's equations of motion for a spinning top}

\begin{equation*}
\begin{aligned}
I_1\dot{\omega}_1&+(I_3-I_2)\omega_3\omega_2=M_1\\
I_2\dot{\omega}_2&+(I_1-I_3)\omega_1\omega_3=M_2\\
I_3\dot{\omega}_3&+(I_2-I_1)\omega_2\omega_1=M_3
\end{aligned}
\end{equation*}

\definitiontable{
$I_i$&i$_{TH}$ principal moment of inertia&in&$\si{\kilo\gram\meter\squared}$\\
$\dot{\omega}_i$&rotational acceleration in i-direction&in&$\si{\radian\per\second\squared}$\\
$M_i$&torque on $\mathcal{B}$ in i-direction&in&$\si{\newton\meter}=\si{\kilo\gram\meter\squared\per\second\squared}$
}

\begin{itemize}
\item The Euler equations are nothing else than the AMP applied to a fixed reference point or the center of mass of a body. A rotating, principal frame is used.
\item As a result, they are only valid for a fixed point or the center of mass of a body.
\item They can only be applied if a body fixed, principal frame is used.
\item You can modify the equations to get a general formulation if the angular velocity of the coordinate frame is not equal to the angular velocity of the body.
\end{itemize}

\subsection{TSP-rule}

\begin{equation*}
\vec{\omega}_\varphi\times\vec{\omega}_\psi\approx\frac{1}{I_3}\vec{M}_O^{ext}
\end{equation*}

\definitiontable{
$\vec{\omega}_\varphi$&precession&in&$\si{\radian\per\second}$\\
$\vec{\omega}_\psi$&spin&in&$\si{\radian\per\second}$\\
$\vec{M}_O^{ext}$&external torque acting on gyroscope&in&$\si{\newton\meter}=\si{\kilo\gram\meter\squared\per\second\squared}$
}

\begin{itemize}
\item The TSP-Rule holds true for a fast spinning gyroscope.
\item The external torque is aligned with the middle finger.
\item Spin is aligned with the index finger.
\item Precession axis is aligned with the thumb.
\end{itemize}

\section{Vibrations}

\subsection{Equation of motion of a dampened, forced one degree of freedom oscillator}

\begin{equation*}
\ddot{x}+2\delta\dot{x}+\omega_0^2x=f_0(t)
\end{equation*}

where

\begin{equation*}
\delta=\frac{c}{2m}\qquad \omega_0=\sqrt{\frac{k}{m}}
\end{equation*}

\definitiontable{
$c$&damper constant&in&$\si{\kilo\gram\per\second}$\\
$m$&vibrating mass&in&$\si{\kilo\gram}$\\
$\delta$&characteristic damping&in&$\si{\per\second}$\\
$k$&spring constant&in&$\si{\kilo\gram\per\second\squared}$\\
$\omega_0$&natural frequency&in&$\si{\radian\per\second}$
}

\subsection{General solution for free vibrations (one DOF) with distinct roots $\lambda_1\neq\lambda_2$}

The solution in the time domain is

\begin{equation*}
x(t)=A_1e^{\lambda_1 t}+A_2e^{\lambda_2 t}
\end{equation*}

while in the phase space is

\begin{equation*}
\vecd{x}(t)=A_1\vec{s}_1e^{\lambda_1 t}+A_2\vec{s}_2e^{\lambda_2 t}
\end{equation*}

where $\lambda_{1,2}=-\delta\pm\sqrt{\delta^2-\omega_0^2}$ and $\vec{s}_j=(1,\lambda_j)$

\definitiontable{
$\delta$&characteristic damping&in&$\si{\per\second}$\\
$\omega_0$&natural frequency&in&$\si{\radian\per\second}$
}

\subsection{General solution for free vibrations (one DOF) with repeated roots $\lambda= \lambda_1=\lambda_2$}
The solution in the time domain is

\begin{equation*}
x(t)=A_1e^{\lambda t}+A_2te^{\lambda t}
\end{equation*}

while in the phase space is

\begin{equation*}
\vecd{x}(t)=(A_1\vec{s}_1+A_2\highlight{\tilde{\vec{s}}_1})e^{\lambda t}+A_2\vec{s}_2te^{\lambda t}
\end{equation*}

where $\lambda_{1,2}=-\delta = -\omega_0$, $\vec{s}_j=(1,\lambda)$ and $\highlight{\tilde{\vec{s}}_1=(0,1)}$

\definitiontable{
$\delta$&characteristic damping&in&$\si{\per\second}$\\
$\omega_0$&natural frequency&in&$\si{\radian\per\second}$
}

\subsection{Forced harmonic response}

\begin{equation*}
\ddot{x}+2\delta\dot{x}+\omega_0^2x=f_0\sin(\omega t)
\end{equation*}

the steady state solution is

\begin{equation*}
x_p(t)=A\sin(\omega t-\phi)
\end{equation*}

where

\begin{equation*}
A=\frac{f_0}{\sqrt{(\omega_0^2-\omega^2)+4\delta\omega^2}}\quad\text{and}\quad\phi=\arctan\left(\frac{2\delta \omega}{\omega_0^2-\omega^2}\right)
\end{equation*}

or in the form

\begin{equation*}
x_p(t)=\frac{f_0}{\omega_0^2}V(\eta,D)\sin(\omega t-\phi(\eta, D))
\end{equation*}

where 

\begin{equation*}
V=\frac{1}{\sqrt{(1-\eta^2)^2+(2D\eta)^2}}\quad\text{and}\quad\phi=\arctan\left(\frac{2D\eta}{1-\eta^2}\right)
\end{equation*}

and

\begin{equation*}
\eta=\frac{\omega}{\omega_0}\quad\text{and}\quad D=\frac{\delta}{\omega_0}
\end{equation*}

\definitiontable{
$\delta$&characteristic damping&in&$\si{\per\second}$\\
$\omega_0$&natural frequency&in&$\si{\radian\per\second}$\\
$\omega$&rotational velocity&in&$\si{\radian\per\second}$\\
$\eta$&frequency ratio\\
$D$&Lehr's damping
}

\myspic{0.5}{ForcedVibrations}

\subsection{Equation for vibrations with $n$ degrees of freedom}

\begin{equation*}
\tvec{M}\vecdd{x}+\tvec{C}\vecd{x}+\tvec{K}\vec{x}=\vec{F}(t)
\end{equation*}

\definitiontable{
$\vec{x}(t)$&deviation from the stable equilibrium&in&$\si{\meter}$\\
$\tvec{M}$&positive definite, symmetric mass matrix&in&$\si{\kilo\gram}$\\
$\tvec{C}$&positive semidefinite, symmetric damping matrix&in&$\si{\kilo\gram\per\second}$\\
$\tvec{K}$&positive semidefinite, symmetric stiffness matrix&in&$\si{\kilo\gram\per\second\squared}$\\
$\vec{F}(t)$&external force acting on the system&in&$\si{\kilo\gram\meter\per\second\squared}$
}

\subsection{Structural damping}

\begin{equation*}
\tvec{C}=a\tvec{M}+b\tvec{K}
\end{equation*}

\subsection{Generalized eigenvalue problem}

\begin{equation*}
(\lambda^2\tvec{M}+\tvec{K})\vec{u}=\vec{0}
\end{equation*}

\subsection{Purely imaginary eigenvalue pairs}

$\lambda=\pm\ i\ \omega_j,\ \omega_j>0,\ \vec{u}_j\in\mathbb{R}^n$

The solution in the time domain is

\begin{equation*}
\vec{x}_j(t)=\vec{u}_jA_j\sin(\omega_j t+\phi_j)
\end{equation*}

\subsection{Zero eigenvalue pairs} 

$ \lambda = 0^2,\ \vec{u}\in\mathbb{R}^n $

The solution in the time domain is

\begin{equation*}
\vec{x}_j(t)=u_j(B_j+C_jt)
\end{equation*}

\subsection{Modal decomposition of free, undampened vibrations}

\begin{align}
\tvec{U}^T\tvec{M}\tvec{U}&=\diag(m_1,m_2,\ldots,m_n)\\
\tvec{U}^T\tvec{K}\tvec{U}&=\diag(k_1,k_2,\ldots,k_n)
\end{align}

\subsection{Modal equation of motion}

\begin{equation*}
m_j\ddot{y}_j+c_j\dot{y}_j+k_jy_j=f_j(t),\ j=1,\ldots n\qquad c_j=am_j+bk_j\geq 0
\end{equation*}

\subsection{Forced response to harmonic forcing of a $n$ degrees of freedom system}

For harmonic forcing

\begin{equation*}
\vec{F}(t)=\vec{F}_0\sin(\omega t)
\end{equation*}

the steady-state solution of a generic modal amplitude $[y_P(t)]_j$ is given by

\begin{equation*}
[y_P(t)]_j=A_j^P\sin(\omega t+\phi_j)
\end{equation*}

with

\begin{equation*}
A_j^P=\frac{f_j^0}{\sqrt{(\omega_j^2-\omega^2)^2+c_j^2\omega^2}},\quad \phi_j=\arctan\left(\frac{c_j\omega}{\omega_j^2-\omega^2}\right)\quad\text{and}\quad f_j^0=(\tvec{U}^T\vec{F}_0)_j
\end{equation*}

\begin{itemize}
\item Note that this solution is the particular solution to $\tvec{M}\vecdd{x}+\tvec{C}\vecd{x}+\tvec{K}\vec{x}=\vec{F}(t)$, transformed to modal coordinates such that the system is decoupled.

Like that the system is described by a series of one DOF oscillators, one for each modal coordinate. For those the solutions for forced, damped vibrations are applicable again.
\item Each mode has its resonance frequency $\omega_j = \sqrt{\frac{k_j}{m_j}}$.
\item When the external forcing is given as above we only care about the particular solution as given above, since the transient response defined by the homogeneous solution decays after some time.
\item Remember that from modal coordinates one needs to transform back to the original $\vec{x}$-coordinate frame.

\begin{equation*}
\vec{x}_p = \tvec{U}\vec{y}_p(t) = \begin{pmatrix}\vec{u}_1&\vec{u}_1&\cdots&\vec{u}_n\end{pmatrix}\begin{pmatrix}
A_1^P\sin(\omega t+\phi_1\\\vdots\\ A_n^P\sin(\omega t+\phi_n)
\end{pmatrix}
\end{equation*} 

which can be written as:

\begin{equation*}
\vec{x}_p(t)=\sum\limits_{k=1}^n A_k^P\vec{u}_k\sin(\omega t+\phi_k)
\end{equation*}

Based on that an amplification factor can be established where $B_j(\omega) =\sum\limits_{k=1}^n A_k^P(\vec{u}_k)_j$ is the amplitude of the particular response of the $j^{th}$ degree of freedom.

\begin{equation*}
V_j(\omega) = \frac{|B_j|}{|B_j(0)|}
\end{equation*}

That is the relation of the amplification at any frequency $\omega$ w.r.t. the amplification at a frequency of $\omega = 0$.

\myspic{0.6}{AmplificationFactor}

\item As seen in figure \ref{fig:Ampl} the amplification factor has maxima at each of the modal resonance frequencies and for the example shown here it has anti resonances between the maxima. However this depends on the underlying system. See Exercise 13.3 for a counterexample.

\end{itemize}

\section{Analytic Dynamics}
\subsubsection{Generalized Coordinates}
The generalized coordinates are variables required to define the position of a body uniquely. They are not unique themselves.
\subsubsection{Constraints}

\begin{define}
A \textbf{constraint} is a scalar function of generalized coordinates, generalized velocities and time. This constraint is \textbf{non-holonomic}.
\textcolor{red}{Also accelerations, and higher derivatives?}

\important{$f(\vec{q},\vecd{q},t)=0$}
\end{define}

\begin{define}
A \textbf{sceleronomic} constraint is only a function of generalized coordinates.

\important{$f(\vec{q})=0$}
\end{define}

\begin{define}
A \textbf{rheonomic} constraint is only a function of generalized coordinates and time.

\important{$f(\vec{q},t)$}
\end{define}

\begin{define}
If non-holonomic constraints can be integrated, they will become holonomic and also eliminate DoFs.

Every holonomic constraint can be given in \textbf{Pfaffian form} (exact differential):

\important{$f(\vec{q},t)=0\quad\rightarrow\quad df=\sum\limits_{i=1}^n\frac{\partial f(\vec{q},t)}{\partial q_i}dq_i+\frac{\partial f(\vec{q},t)}{\partial t}dt=0$}
\end{define}

\begin{define}
It can also be given in \textbf{velocity form} by dividing by $dt$:

\important{$\frac{df}{dt}=\sum\limits_{i=1}^n\frac{\partial f(\vec{q},t)}{\partial \dot{q}_i}+\frac{\partial f(\vec{q},t)}{\partial t}=0$}
\end{define}

\begin{define}
In practice many constraints are given in a \textbf{linear velocity form}:

\important{$\sum\limits_{i=1}^na_i(\vec{q},t)\dot{q}_i+b(\vec{q},t)=0$}
\end{define}

\subsubsection{Showing that a constraint is holonomic}

A constraint in \textbf{velocity form} is holonomic if its coefficients $a_i$ and $b_i$ can be mapped to a \textbf{pfaffian form}. If so then the velocity form is an exact differential

\mportant{$\frac{df}{dt}=\sum\limits_{i=1}^na_i(\vec{q},t)\dot{q}_i+b(\vec{q},t)$}

What remains to be showed is that:

\mportant{$\frac{\partial f(\vec{q},t)}{\partial q_i}=Ca_i\quad,\quad \frac{\partial f(\vec{q},t)}{\partial t}=Cb,\quad\text{where }C=C(\vec{q},t)$}

The condition for partial derivatives to be integrable

\mportant{$\frac{\partial^2f(\vec{q},t)}{\partial q_k\partial q_i}=\frac{\partial^2f(\vec{q},t)}{\partial q_i\partial q_k}\Rightarrow\frac{\partial C a_i}{\partial q_k}=\frac{\partial C a_k}{\partial q_i}\quad;\quad \frac{\partial Cb}{\partial q_k}=\frac{\partial Ca_i}{\partial t}$}

If $C=0$ the integration constant does not exist, the trivial solution of the problem is found. Thus the constraint is non-holonomic.

\subsubsection{Example: Rolling without slipping in 3D}

\myspic{0.5}{RollingWithoutSlipping}

First we need to define a set of generalized coordinates:

\mportant{$\vec{q} = \begin{bmatrix}
x_C&y_C&\varphi&\alpha
\end{bmatrix}$}

Then the rolling without slipping constraint is defined:

\mportant{$\begin{matrix}
f_1(\vec{q},\vecd{q})&\dot{x}_C-R\dot{\varphi}\sin\alpha=0\\
f_2(\vec{q},\vecd{q})&\dot{y}_C-R\dot{\varphi}\cos\alpha=0
\end{matrix}$}

In linear velocity form:

\mportant{$\begin{matrix}
a_{11}\dot{x}_C+a_{12}\dot{y}_C+a_{13}\dot{\varphi}+a_{14}\dot{\alpha}+b_1=0\\
a_{21}\dot{x}_C+a_{22}\dot{y}_C+a_{23}\dot{\varphi}+a_{24}\dot{\alpha}+b_2=0
\end{matrix}$}

where the coefficients are given by:

\mportant{$\begin{matrix}
a_{11} = 1;&a_{12} = 0;&a_{13}=-R\sin\alpha;&a_{14}=0;&b_1=0\\
a_{21} = 0;&a_{22} = 1;&a_{13}=-R\cos\alpha;&a_{14}=0;&b_2=0
\end{matrix}$}

Now if there exists a $C(\vec{q},t)$ that fulfils for the partial derivatives to be integrable, the considered constraints are holonomic. Therefore we evaluate $\pfrac{Ca_i}{q_k}=\pfrac{Ca_k}{q_i}$ to find $C(\vec{q},t)$:

For $f_1(\vec{q},t)$:

\mportant{$a_{12}=0\Rightarrow \pfrac{Ca_{12}}{x_C}=\pfrac{Ca_{11}}{y_C} = 0\Rightarrow \pfrac{C}{y_C}=0$}

For which reason $C(\vec{q},t)$ is independent of $y_C$.

Similarly:

\mportant{$a_{14}=0\Rightarrow \pfrac{Ca_{14}}{x_C}=\pfrac{Ca_{11}}{\alpha}=0\Rightarrow\pfrac{C}{\alpha}=0$}

Therefore $C(\vec{q},t)$ is independent of $\alpha$.

Now since $a_{14}=0$ we can say that:

\mportant{$\pfrac{C(x_C,\varphi)R\sin\alpha}{\alpha}=\pfrac{C(x_c,\varphi)a_{14}}{\varphi}=0$}

But $\pfrac{C(x_C,\varphi)R\sin\alpha}{\alpha}= C(x_C,\varphi)R\cos\alpha\neq0$ unless $C(x_C,\varphi)=0$

\textbf{Therefore this constraint is not holonomic!}

\subsection{Virtual displacements}
\subsubsection{Sceleronomic constraint}
\importname{Actual path}{$\hat{\vec{r}}(t)=q_1\hat{\vec{e}}_1+\cdots+q_n\hat{\vec{e}}_n$}
\importname{Variational path}{$\tilde{\vec{r}}(t)=\tilde{q}_1\hat{\vec{e}}_1+\cdots+\tilde{q}_n\hat{\vec{e}}_n$}
\importname{Virtual displacement}{$\delta\hat{\vec{r}}=\tilde{\vec{r}}-\hat{\vec{r}}=\delta q_1\hat{\vec{e}}_1+\cdots+\delta q_n\hat{\vec{e}}_n$}
\begin{itemize}
\item The virtual displacement is the difference between the actual and the virtual path.
\item It is equivalent to freezing time and moving to a possible, close path.
\item Possible means that all constraints are satisfied, i.e. $\tilde{\vec{r}}(t)$ and $\hat{\vec{r}}(t)$  must lie on $f(\vec{q})=0$.
\importname{Infinitesimal displacement}{$d\hat{\vec{r}}=\frac{\partial \hat{\vec{r}}}{\partial q_1}dq_1+\cdots+\frac{\partial\hat{\vec{r}}}{\partial q_n}dq_n$}
\importname{Generalized velocity}{$\hat{\vec{v}}=\frac{\partial\hat{\vec{r}}}{\partial q_1}\dot{q}_1+\cdots+\frac{\partial\hat{\vec{r}}}{\partial q_n}\dot{q}_n$}

%\begin{itemize}
\item $d\hat{\vec{r}}$ is tangent to the trajectory, $\delta\hat{\vec{r}}$ is not.
\end{itemize}

\importname{Normal vector}{$\vec{n}=\frac{\nabla f}{|\nabla f|}=\frac{1}{|\nabla f|}\left(\frac{\partial f}{\partial q_1}\hat{\vec{e}}_1+\cdots+\frac{\partial f}{\partial q_n}\hat{\vec{e}}_2\right)$}

\begin{itemize}
\item  Both are perpendicular to the gradient of the constraint plane. Thus the following conditions must hold.
\end{itemize}

\begin{align*}
\nabla f\cdot d\hat{\vec{r}}=0&\Rightarrow\frac{\partial f}{\partial q_1}dq_1+\cdots+\frac{\partial f}{\partial q_1}dq_n=0\\
\nabla f\cdot \partial\hat{\vec{r}}=0&\Rightarrow\frac{\partial f}{\partial q_1}\delta q_1+\cdots+\frac{\partial f}{\partial q_n}\delta q_n =0
\end{align*}

\subsubsection{Rheonomic constraint}

\begin{itemize}
\item $d\hat{\vec{r}}$ and $\hat{\vec{v}}$ do not lay on the constraint surface $\nabla f\cdot d\hat{\vec{r}}\neq 0$
\item $\delta\hat{\vec{r}}$ does lay on the tangent plane  $\nabla f\cdot \delta\hat{\vec{r}}=\delta f=0$
\end{itemize}

\subsubsection{Kinematical method to compute virtual displacements}

Recall the definition of velocity:

\important{$\vec{v}=\frac{d\vec{r}}{dt}=\sum\limits_{i=1}^n\textcolor{red}{\frac{\partial\vec{r}}{\partial q_i}}\dot{q}_i+\frac{\partial \vec{r}}{\partial t}$}

and of virtual displacements:

\important{$\delta \vec{r}=\sum\limits_{i=1}^n\textcolor{red}{\frac{\partial\vec{r}}{\partial q_i}}\delta q_i$}

\begin{enumerate}
\item Take the velocity formula.
\item Identify the linear terms in $\dot{q}_i$.
\item Discard the rest.
\item Replace $\dot{q}_i$ with $\delta q_i$.
\end{enumerate}

\textbf{Use this method when only the velocity is known.}

\subsection{D'alemberts principle}
\subsubsection{Reaction forces}
A constraint force keeps the system on the surface $f(\vec{q},t)=0$ in the configuration space.

\importname{Constraint force}{$\vec{R}=R\vec{n}$}

\begin{define}
The \textbf{virtual work} is the work done by a force onto a virtual displacement.
\end{define}

\textbf{The virtual work of a constraint force is zero.}

\begin{define}
An \textbf{ideal constraint} is such that the associated constraint force does not do work along virtual displacements compatible with the constraint.
\end{define}
\subsubsection{Projected Newton's equations}
\begin{enumerate}
\item Assume a system of $N$ particles of mass $m_i$, subjected to holonomic constraints. The LMP writes:

\mportant{$m_i\ddot{\vec{r}}_i=\vec{F}_i+\vec{R}_i$}

\item Choose generalized coordinates $\vec{q}$ compatible with the constraints \emph{(admissible)}:

\mportant{$\vec{r}_i=\vec{r}_i(\vec{q},t)$}

and the corresponding virtual displacement:

\mportant{$\delta\vec{r}_i=\frac{\partial\vec{r}_i}{\partial \vec{q}}\delta\vec{q}$}
\item Bring all terms of the LMP to the LHS \emph{(dynamic equilibrium)}

\mportant{$m_i\ddot{\vec{r}}_i-\vec{F}_i-\vec{R}_i=\vec{0}$}
\item Project the dynamic equilibrium onto the virtual displacement:

\mportant{$\sum\limits_{i=1}^N\delta \vec{r}_i\cdot(m_i\vecdd{r}_i-\vec{F}_i-\vec{R}_i)=0$}

\item Reaction forces do not do work on an admissible virtual displacement:

\mportant{$\sum\limits_{i=1}^N\delta\vec{r}_i\cdot\vec{R}_i=0$}
\end{enumerate}

\importname{D'alembers principle}{$\sum\limits_{i=1}^N\delta\vec{r}_i\cdot(m_i\vecdd{r}_i-\vec{F}_i)=0$}

\begin{enumerate}\addtocounter{enumi}{4}
\item Recall the definition of the virtual displacements

\mportant{$\delta\vec{r}_i=\frac{\partial\vec{r}_i}{\partial\vec{q}}\delta\vec{q}=\sum\limits_{k=1}^p\frac{\partial\vec{r}_i}{\partial q_k}\delta q_k$}
\item And insert it into d'alemberts principle
\mportant{$\sum\limits_{i=1}^N(m_i\vecdd{r}_i-\vec{F}_i)\cdot\sum\limits_{k=1}^p\frac{\partial\vec{r}_i}{\partial q_k}\delta q_k=0$}
\item Since all general coordinates can be varied independently (admissible) we can write $p$ independent equations:
\important{$\sum\limits_{i=1}^N(m_i\vecdd{r}_i-\vec{F}_i)\cdot\frac{\partial \vec{r}_i}{\partial q_k}=0,\quad \forall k=1,\cdots ,p$}
\end{enumerate}
\subsubsection{Generalized forces}
\begin{define}
The projection of the active forces onto the admissible direction dictated by the variation of $q_k$ is called \textbf{generalized force}.

\important{$Q_k=\sum\limits_{i=1}^N\vec{F}_i\cdot\frac{\partial \vec{r}_i}{\partial q_k}$}
\end{define}
\subsection{Lagrange equations}
\begin{define}
Consider a system of $N$ particles. The \textbf{kinetic energy} is defined as:

\important{$\mathcal{T}=\sum\limits_{i=1}^N\onha m_i\vecd{r}_i\cdot\vecd{r}_i=\sum\limits_{i=1}^N\onha m_i|\vecd{r}_i|^2$}
\end{define}

Note that:

\mportant{$\ddt\left(\frac{\partial\mathcal{T}}{\partial \vecd{r}_i}\right)=m_i\vecdd{r}_i$}

And therefore d'Alembert's priciple can be written as:

\mportant{$\sum\limits_{i=1}^N\left[\ddt\left(\frac{\partial\mathcal{R}}{\partial\vecd{r}_i}\right)-\vec{F}_i\right]\cdot\delta\vec{r}_i=0\Rightarrow\sum\limits_{k=1}^p\sum\limits_{i=1}^N\left[\ddt\left(\frac{\partial\mathcal{T}}{\partial\vecd{r}_i}\right)-\vec{F}_i\right]\cdot\frac{\partial \vec{r}_i}{\partial q_k}\delta q_k=0$}

If the GC's are kinematically admissible, then

\mportant{$\sum\limits_{i=1}^N\left[\ddt\left(\frac{\partial\mathcal{T}}{\partial \vecd{r}_i}\right)-\vec{F}_i\right]\cdot\frac{\partial\vec{r}_i}{\partial q_k}=0\quad k=1,\ldots,p$}

Note that:

\mportant{$\ddt\left(\frac{\partial\mathcal{T}}{\partial\vecd{r}_i}\cdot\frac{\partial\vec{r}_i}{\partial q_k}\right)=\ddt\left(\frac{\partial\mathcal{T}}{\partial\vecd{r}_i}\cdot\frac{\partial\vec{r}_i}{\partial q_k}\right)+\frac{\partial\mathcal{T}}{\partial\vecd{r}_i}\cdot\ddt\left(\frac{\partial\vec{r}_i}{\partial q_k}\right)$}

and therefore:

\mportant{$\ddt\left(\frac{\partial\mathcal{T}}{\partial\vecd{r}}\right)\cdot\frac{\partial \vec{r}_i}{\partial q_k}=-\frac{\partial\mathcal{T}}{\partial\vecd{r}_i}\cdot\ddt\left(\frac{\partial\vec{r}_i}{\partial q_k}\right)+\ddt\left(\frac{\partial \mathcal{T}}{\partial\vecd{r}_i}\cdot\frac{\partial\vec{r}_i}{\partial q_k}\right)$}

Substitution yields:

\mportant{$\sum\limits_{i=1}^N\left[-\frac{\partial\mathcal{T}}{\partial\vecd{r}_i}\cdot\textcolor{red}{\ddt\left(\frac{\partial\vec{r}_i}{\partial q_k}\right)}+\ddt\left(\textcolor{blue}{\frac{\partial\mathcal{T}}{\partial\vecd{r}_i}\cdot\frac{\partial\vec{r}_i}{\partial q_k}}\right)-\vec{F}_i\cdot\frac{\partial\vec{r}_i}{\partial q_k}\right]=0\quad k=1,\ldots,p$}

Now consider \textcolor{red}{red}:

\mportant{$\frac{\partial\mathcal{T}}{\partial\vecd{r}_i}\cdot\ddt\left(\frac{\partial\vec{r}_i}{\partial q_k}\right)=\frac{\partial\mathcal{T}}{\partial\vecd{r}_i}\cdot\frac{\partial\vecd{r}_i}{\partial q_k}=\frac{\partial\mathcal{T}}{\partial q_k}$}

Then consider \textcolor{blue}{blue}:

\mportant{$\vecd{r}_i=\sum\limits_{i=1}^p\frac{\partial\vec{r}_i}{\partial q_k}\dot{q}_k+\frac{\partial \vec{r}_i}{\partial t}\quad\Rightarrow\quad \frac{\partial\vecd{r}_i}{\partial\dot{q}_k}=\frac{\partial\vec{r}_i}{\partial q_k}$}

and therefore

\mportant{$\textcolor{blue}{\frac{\partial\mathcal{T}}{\partial \vecd{r}_i}\cdot\frac{\partial\vec{r}_i}{\partial q_k}}=\frac{\partial\mathcal{T}}{\partial\vecd{r}_i}\cdot\frac{\partial\vecd{r}_i}{\partial \dot{q}_i}=\frac{\partial \mathcal{T}}{\partial q_k}$}

\important{$\ddt\left(\frac{\partial\mathcal{T}}{\partial\dot{q}_k}\right)-\frac{\partial\mathcal{T}}{\partial q_k}-Q_k=0,\quad k=1,\ldots,p$}

\begin{itemize}
\item LEs are nothing else than d'Alembert's principle, written in terms of kinetic energies.
\item As such, the chosen generalized coordinates are supposed to identically satisfy the constraints (i.e. the are \textbf{unconstrained coordinates}.
\item By working with energy, the inertial terms involve less algebra than the d'Alembert's principle.
\end{itemize}

\subsubsection{Potential active forces}

If all active forces are potential:

\mportant{$\vec{F}_i=-\frac{\partial\mathcal{V}}{\partial\vec{r}_i}+\vec{F}_i^{nc}$}

Therefore the generalized forces become:

\mportant{$Q_k=\sum\limits_{i=1}^N\vec{F}_i\cdot\frac{\partial\vec{r}_i}{\partial q_k}=-\sum\limits_{i=1}^N\left[\frac{\partial\mathcal{V}}{\partial\vec{r}_i}\cdot\frac{\partial\vec{r}_i}{\partial q_k}+\vec{F}_i^{nc}\cdot\frac{\partial \vec{r}_i}{\partial q_k}\right]=-\frac{\partial\mathcal{V}}{\partial q_k}+Q_k^{nc}$}

And then the LEs could be written as:

\important{$\ddt\left(\frac{\partial\mathcal{T}}{\partial \dot{q}_k}\right)-\frac{\partial\mathcal{T}}{\partial q_k}+\frac{\partial\mathcal{V}}{\partial q_k}-Q_k^{nc}=0,\quad k=1,\ldots,p$}

\important{$\ddt\left(\frac{\partial\mathcal{T}}{\partial\dot{\vec{q}}}\right)-\frac{\partial\mathcal{T}}{\partial\vec{q}}+\frac{\partial\mathcal{V}}{\partial\vec{q}}=0$}

\subsubsection{Example: Equation of motion of a 2D rigid body: D'alembert's principle}

\myspic{0.5}{Dalembert2D}

\begin{enumerate}
\item $\vec{r}_k = \vec{r}_P+\vec{r}_{Pk}(\theta)$
\item Virtual displacement: $\delta\vec{r}_k =\delta\vec{r}_P+\delta\vec{r}_{Pk}(\theta)$
\item Obtain the virtual displacement $\delta \vec{r}_{Pk}$: Take the variation of the rigid body constraint:

$\vec{r}_{Pk}\cdot\vec{r}_{Pk}=const\Rightarrow 2\delta\vec{r}_{Pk}\cdot\vec{r}_{Pk}=0$

Thus $\delta\vec{r}_{Pk}$ has to be orthogonal to $\vec{r}_{Pk}$.
\item The LMP of each particle: $m_k\vecdd{r}_k-\vec{F}_k-\vec{R}_k = 0$
\item Projected onto the virtual displacement: 

$\sum\limits_{k=1}^N\delta\vec{r}_k\cdot(m_k\vecdd{r}_k-\vec{F}_k-\vec{R}_k)=\vec{0}\qquad \sum\limits_{k=1}^N\delta\vec{r}_k\cdot\vec{R}_k = 0$
\item Substitution yields: $\sum\limits_{k=1}^N\left(\delta\vec{r}_P+\delta\theta\vec{e}_z\times\vec{r}_{Pk}\right)\cdot\left(m_k\vecdd{r}_k-\vec{F}_k\right)=0$
\item Since the virtual displacements are arbitrary we can write 2 different equations \textcolor{red}{why is arbitrariness necessary?}

\begin{align*}
\sum\limits_{k=1}^N\delta\vec{r}_P\cdot\left(m_k\vecdd{r}_k-\vec{F}_k\right)&=0\\
\sum\limits_{k=1}^N\left(\delta\theta\vec{e}_z\times\vec{r}_{Pk}\right)\cdot\left(m_k\vecdd{r}_k-\vec{F}_k\right)&=0
\end{align*}

\important{$\vec{a}\cdot(\vprod{b}{c})=\vec{b}\cdot(\vprod{c}{a})=\vec{c}\cdot(\vprod{a}{b})$}

\item
$\sum\limits_{k=1}^N\vec{r}_{Pk}\times\left(m_k\vecdd{r}_k-\vec{F}_k\right)\cdot\delta\theta\vec{e}_z=0$
\item Which results in the equations:

\begin{align*}
\sum\limits_{k=1}^Nm_k\vecdd{r}_k&=\sum\limits_{k=1}^N\vec{F}_k\\
\sum\limits_{k=1}^Nm_k\vec{r}_{Pk}\times\vecdd{r}_k&=\sum\limits_{k=1}^N\vec{r}_{Pk}\times\vec{F_k}
\end{align*}
\item Now these equations need to be represented in terms of the general coordinates $\vec{r}_P$ and $\theta$:

\begin{align*}
\vecd{r}_k&=\vecd{r}_P+\dot{\theta}\vec{e}_z\times\vec{r}_{Pk}\\
\vecdd{r}_k&=\vecdd{r}_P+\ddot{\theta}\vec{e}_z+\dot{\theta}\vec{e}_z\times\vec{r}_{Pk}\\
&=\vecdd{r}_P+\ddot{\theta}\vec{e}_z\times\vec{r}_{Pk}+\dot{\theta}\vec{e}_z\times\left(\dot{\theta}\vec{e}_z\times\vec{r}_{Pk}\right)\\
&=\vecdd{r}_P+\ddot{\theta}\vec{e}_z\times\vec{r}_{Pk}-\dot{\theta}^2\vec{r}_{Pk}
\end{align*}

\mportant{$\vec{r}_{PC}=\frac{1}{m_{tot}}\sum\limits_{k=1}^Nm_k\vec{r}_{Pk},\qquad m_{tot} =\sum\limits_{k=1}^Nm_k$}
\item 
\begin{align*}
m_{tot}\left(\vecdd{r}_P+\ddot{\theta}\vec{e}_z\times\vec{r}_{PC}-\dot{\theta}^2\vec{r}_{PC}\right)&=\sum\limits_{k=1}^N\vec{F}_k=\vec{F}\\
m_{tot}\vec{r}_{PC}\times\vecdd{r}_P+\sum\limits_{i=1}^Nm_k\vec{r}_{Pk}\times\left(\vec{e}_z\times\vec{r}_{Pk}\right)\ddot{\theta}&=\sum\limits_{i=1}^N\vec{r}_{Pk}\times\vec{F}_k=\vec{M}_P
\end{align*}

Note that:

\mportant{$\sum\limits_{k=1}^Nm_k\vprod{\vec{r}_{Pk}}{\left(\vprod{\vec{e}_z}{\vec{r}_{Pk}}\right)}=\sum\limits_{k=1}^Nm_k\left|\vec{r}_{Pk}\right|^2\vec{e}_z=I_P\vec{e}_z$}
\item The equations of motion for the 2D rigid body become:

\begin{align*}
m_{tot}\left(\vecdd{r}_P+\ddot{\theta}\vec{e}_z\times\vec{r}_{PC}-\dot{\theta}^2\vec{r}_{PC}\right)&=\vec{F}\\
m_{tot}\vec{r}_{PC}\times\vecdd{r}_P+I_P\ddot{\theta}\vec{e}_z=\vec{M}_P
\end{align*}

\item Special cases : $\ddt\vec{r}_P\equiv 0$ und $P=C\quad\Rightarrow\quad \vec{r}_{PC}\equiv 0$
\end{enumerate}

\subsubsection{Equation of motion of a 2D rigid body: Lagrange equations}
\begin{enumerate}
\item 

\begin{align*}
\mathcal{T}&=\onha\sum\limits_{k=1}^Nm_k\vecd{r}_k\cdot\vecd{r}_k\\
&=\onha\sum\limits_{k=1}^Nm_k\left(\vecd{r}_P+\dot{\theta}\vec{e}_z\times\vec{r}_{Pk}\right)\cdot\left(\vecd{r}_P+\dot{\theta}\vec{e}_z\times\vec{r}_{Pk}\right)\\
&=\onha\sum\limits_{k=1}^Nm_k\vecd{r}_P\cdot\vecd{r}_p+\onha\sum\limits_{k=1}^Nm_k\left(\dot{\theta}\vec{e}_z\times\vec{r}_{Pk}\right)\cdot\left(\dot{\theta}\times\vec{r}_{Pk}\right)+\sum\limits_{k=1}^Nm_k\vecd{r}_P\cdot\left(\dot{\theta}\vec{e}_z\times\vec{r}_{Pk}\right)\\
&=\onha m_{tot}\vecd{r}_P\cdot\vecd{r}_P+\onha I_P\dot{\theta}^2+\underbrace{m_{tot}\dot{\theta}\vecd{r}_P\cdot\left(\vec{e}_z\times\vec{r}_{PC}\right)}_{\text{Coupling term }P\neq C}
\end{align*}

\item For the translation:

\begin{align*}
\frac{\partial\mathcal{T}}{\partial\vecd{r}_P}&=m_{tot}\vecd{r}_P+m_{tot}\dot{\theta}\vec{e}_z\times\vec{r}_{PC}\\
\ddt\left(\frac{\partial\mathcal{T}}{\partial\vecd{r}_P}\right)&=m_{tot}\vecdd{r}_P+m_{tot}\ddot{\theta}\vec{e}_z\times\vec{r}_{PC}+m_{tot}\dot{\theta}^2\vec{e}_z\times\left(\vec{e}_z\times\vec{r}_{PC}\right)\\
\frac{\partial\mathcal{T}}{\partial\vec{r}_P}&=\vec{0}
\end{align*}
\item For the rotation:

\begin{align*}
\frac{\partial\mathcal{T}}{\partial\dot{\theta}}&=I_P\dot{\theta}+m_{tot}\vecd{r}_P\cdot\left(\vec{e}_z\times\vec{r}_{PC}\right)\\
\ddt\left(\frac{\partial\mathcal{T}}{\partial\dot{\theta}}\right)&=I_P\ddot{\theta}+m_{tot}\vecdd{r}\cdot\left(\vec{e}_z\times\vec{r}_{PC}\right)+m_{tot}\dot{\theta}\vecd{r}_P\cdot\left(\vec{e}_z\times\left(\vec{e}_z\times\vec{r}_{PC}\right)\right)\\
\frac{\partial\mathcal{T}}{\partial\theta}&=m_{tot}\dot{\theta}\vec{r}_P\cdot\left(\vec{e}_z\times\frac{\partial\vec{r}_{PC}}{\partial\theta}\right)=m_{tot}\dot{\theta}\vecd{r}_P\cdot\left(\vec{e}_z\times\left(\vec{e}_z\times\vec{r}_{PC}\right)\right)
\end{align*}
\item The generalized force for the translation DoFs are:

\mportant{$Q_{\vec{r}_P}=\sum\limits_{k=1}^N\vec{F}_k\cdot\frac{\partial\vec{r}_k}{\partial\vec{r}_P}=\sum\limits_{k=1}^N\vec{F}_k=\vec{F}$}

\item And for rotation:

\mportant{$Q_{\theta}=\sum\limits_{k=1}^N\vec{F}_i\cdot\frac{\partial\vec{r}_k}{\partial\theta}=\sum\limits_{k=1}^N\vec{F}_k\cdot\left(\vec{e}_z\times\vec{r}_{Pk}\right)=\sum\limits_{k=1}^N\vec{e}_z\left(\vec{r}_{Pk}\times\vec{F}_k\right)=\vec{e}_z\cdot\vec{M}_P$}
\item Carrying out the Lagrange equations yields the same equations as for the D'alembert's principle.
\end{enumerate}

\subsubsection{Lagrange equations for a 3D rigid body}



\subsubsection{Lagrange equations for constrained coordinates}

\begin{itemize}
\item Complicated constraint equations
\item Active forces depend on constraint forces
\item Non-holonomic constraints
\end{itemize}

lead to the necesity of choosing GC, which do not fully satisfy the constraints, such that the constraint forces do appear in the EoM.

In tat case the virtual displacement does not lay on the tangent to the constraint surface:

\mportant{$\delta\hat{\vec{r}}\cdot\vec{n}\neq0$}

Thus the mapping between GCs and positions does not identically satisfy the $m$ constraints applied to the system and the constraints have to be enforced!

\mportant{$m_i\vecdd{r}_k-\vec{F}_k-\vec{R}_k=\vec{0}$}

The virtual work done on the virtual displacement is:

\mportant{$\sum\limits_{k=1}^N\delta\vec{r}_k\cdot\left(m_k\vecdd{r}_k-\vec{F}_k-\vec{R}_k\right)=0$}

The work of the reaction forces does not vanish!

What is known is the direction of the reaction force, which is aligned with the gradient of the constraint.

\begin{align*}
f_i(\vec{q},t)&=0,\quad i=1,\ldots,m\\
\vec{r}_k&=\vec{r}_k(\vec{q},t),\quad k=1,\ldots,N\\
\vec{R}_k^{(i)}=\lambda_i\frac{\partial f_i}{\partial\vec{r}_k}
\end{align*}

The constraint force on particle $k$ due to constraint $i$, where $\lambda_i$ is proportional to the magnitude of the scalar force, but is not the magnitude of the constraint force since $\left|\frac{\partial f_i}{\partial\vec{r}_k}\right|\neq 1$!

The total constraint force is then:

\mportant{$\vec{R}_k=\sum\limits_{i=1}^m\vec{R}_k^{(i)}=\sum\limits_{i=1}^m\lambda_i\frac{\partial f_i}{\partial\vec{r}_k}$}

The virtual work of the constraint force does not vanish and the constraint forces cannot be eliminated!

\begin{align*}
\sum\limits_{k=1}^N\vec{R}_k\cdot\delta\vec{r}_k=\sum\limits_{k=1}^N\vec{R}_k\cdot\frac{\partial\vec{r}_k}{\partial\vec{q}}\delta\vec{q}=\sum\limits_{k=1}^N\sum\limits_{i=1}^m\lambda_i\frac{\partial f_i}{\partial\vec{r}_k}\cdot\frac{\partial\vec{r}_k}{\partial \vec{q}}\delta\vec{q}\\
=\sum\limits_{i=1}^m\lambda_i\frac{\partial f_i}{\partial\vec{q}}\delta\vec{q}=\sum\limits_{j=1}^n\sum\limits_{i=1}^m\lambda_i\frac{\partial f_i}{\partial q_j}\delta q_j
\end{align*}

Therefore the LEs gain an additional term representing the generalized constraint forces:

\important{$\ddt\left(\frac{\partial\mathcal{T}}{\partial\dot{q}_j}\right)-\frac{\partial\mathcal{T}}{\partial q_j}+\frac{\partial\mathcal{V}}{\partial q_j}+Q_j^{nc}-\sum\limits_{i=1}^m\lambda_i\frac{\partial f_i}{\partial q_j}=0\quad j=1,\ldots,n$}

which must be completed by $m$ constrains:

\mportant{$f_i(\vec{q},t)=0,\quad i=1,\ldots,m$}

\subsubsection{Non-holonomic constraints}

\mportant{$\sum\limits_{i=1}^n a_i(\vec{q},t)\dot{q}_i+b(\vec{q},t)=0$}

When holding time fixed:

\mportant{$\sum\limits_{i=1}^na_i(\vec{q},t)\delta q_i=0$}

Which compactly writes as:

\mportant{$\vec{q}\cdot\delta\vec{q}=0$}

Therefore the reaction force is defined es $\vec{R}=\lambda\vec{a}$

And the LEs write out as:

\begin{align*}
\ddt\left(\frac{\partial\mathcal{T}}{\partial\vecd{q}}\right)-\frac{\partial\mathcal{T}}{\partial\vec{q}}+\frac{\partial\mathcal{V}}{\partial\vec{q}}+\vec{Q}^{nc}-\sum\limits_{i=1}^m\lambda_i\vec{a}_i=\vec{0},\quad\vec{q}\in\mathbb{R}^n\\
\vec{a}_i\cdot\vec{q}_i+\vec{b}_i=0,\quad i=1,\ldots,m
\end{align*}

\subsection{Hamiltons principle}

\mportant{$\sum\limits_{j=1}^n\int_{t_1}^{t_2}\left(\ddt\left(\frac{\partial\mathcal{T}}{\partial\dot{q}_j}\right)-\frac{\partial\mathcal{T}}{\partial q_j}+\frac{\partial\mathcal{V}}{\partial q_j}\right)\delta q_j dt = 0$}

Consider a virtual displacement that vanishes at $t_1$ and $t_2$: \textcolor{red}{What constraints does that impose on $t_1$ and $t_2$?}

\mportant{$\delta \vec{r}_k(t_1)=\delta\vec{r}_k(t_2)=\vec{0}$}

By integrating by part the second term then vanishes as:

\mportant{$\int_{t_1}^{t_2}\ddt\left(\frac{\partial\mathcal{T}}{\partial\dot{q}_j}\right)\delta q_jdt=-\int_{t_1}^{t_2}\frac{\partial\mathcal{T}}{\partial\dot{q}_j}dt$}

Thus the initial equation can be written as:

\mportant{$\sum\limits_{j=1}^n\int_{t_1}^{t_2}\left(-\frac{\partial\mathcal{T}}{\partial \dot{q}_j}\delta\dot{q}_j-\frac{\partial\mathcal{T}}{\partial q_k}\delta q_j+\frac{\partial\mathcal{V}}{\partial q_j}\delta q_j\right)dt=0$}

Note that the first two terms are the variation of the kinetic energy:

\mportant{$\sum\limits_{j=1}\left(\frac{\partial\mathcal{T}}{\partial\dot{q}_j}\delta\dot{q}_j+\frac{\partial\mathcal{T}}{\partial q_j}\delta q_j\right)=\delta \mathcal{T}$}

We finally obtain:

\begin{align*}
\delta\int_{t_1}^{t_2}\left(\mathcal{T}-\mathcal{V}\right)dt=0\\
\delta\vec{q}(t_1)=\delta\vec{q}(t_2)=\vec{0}
\end{align*}

The Hamilton's principle can be stated as: The real trajectory of a system is such that the integral

\important{$\int_{t_1}^{t_2}\left(\mathcal{T}-\mathcal{V}\right)dt$}

stays stationary with respect to any compatible virtual displacement which vanishes at the end of the considered time interval. The integral above is called the \textbf{action}, thus the Hamilon's principle is referred to as the \textbf{principle of stationary action}.

\subsubsection{Example: Pendulum}

\mportant{$\mathcal{L}=\mathcal{T}-\mathcal{V}=\onha mL^2\dot{\theta}^2+mgL\cos\theta$}

The Hamilton's principle then writes:

\mportant{$\delta\int_{t_1}^{t_2}\mathcal{L}dt=\int_{t_1}^{t_2}\delta\mathcal{L}dt=\int_{t_1}^{t_2}\delta\left(\onha mL^2\dot{\theta}^2+mgL\cos\theta\right)dt=0$}

The action becomes:

\mportant{$\int_{t_1}^{t_2}\left(mL^2\dot{\theta}\delta\dot{\theta}-mgL\sin\theta\delta\theta\right)dt=0$}

The equation of motion can not be easily seen at this point since the action still depends on $\dot{\theta}$ and thus setting the integral equal to zero does not simplify it. For that reason an integration by part is made:

\mportant{$\left[mL^2\dot{\theta}\delta\dot{\theta}\right]_{t_1}^{t_2}-\int_{t_1}^{t_2}\left(mL^2\ddot{\theta}+mgL\sin\theta\right)\delta\theta dt = 0$}

Now the integral has to vanish for any arbitrary virtual rotation and thus the argument of the integral has to be zero, which leads to the pendulum equation:

\important{$mL^2\ddot{\theta}+mgL\sin\theta=0$}

\subsubsection{Hamilton's principle for continuous systems}

Let us consider a 1D continuum body of length $l$, laying on the x axis and assume that

\mportant{$\mathcal{T}=\int_0^lT(x,t)dx\quad\mathcal{V}=\int_0^lV(x,t)dx$}

are the kinetic and potential energy densities. The system motion is given by 

\mportant{$v=v(x,t)$}

Assuming the following dependencies:

\mportant{$T=T(v,\dot{v},v',\dot{v}')\quad V=V(v,v',v'')$}

such that the Lagrangian density function acquires the dependency

\mportant{$L=T-V=L(v,v',v'',\dot{v},\dot{v}')$}

The Hamilton's principle then writes 

\important{$\int_{t_1}^{t_2}\int_0^l\delta L dx dt=\int_{t_1}^{t_2}\int_0^l\left[\underbrace{\frac{\partial L}{\partial v}\delta v}_{A}+\underbrace{\frac{\partial L}{\partial v'}\delta v'}_{B}+\underbrace{\frac{\partial L}{\partial v''}\delta v''}_{C}+\underbrace{\frac{\partial L}{\partial\dot{v}}\delta\dot{v}}_{D}+\underbrace{\frac{\partial L}{\partial \dot{v}'}\delta\dot{v}'}_{E}\right]dxdt=0$}

All terms $A-E$ need to be expressed in $\delta v$ only to make integration possible.

\begin{align*}
(B)\quad&\int_0^l\frac{\partial L}{\partial v'}\delta v' dx = \left[\colorboxed{red}{\frac{\partial L}{\partial v'}}\delta v\right]_0^l-\int_0^l\colorboxed{green}{\frac{d}{dx}\left(\frac{\partial L}{\partial v'}\right)}\delta v dx\\
(C)\quad&\int_0^l\frac{\partial L}{\partial v''}\delta v''dx = \left[\frac{\partial L}{\partial v''}\delta v'\right]_0^l-\int_0^l\frac{d}{dx}\left(\frac{\partial L}{\partial v''}\right)\delta v' dx\\
&=\left[\colorboxed{blue}{\frac{\partial L}{\partial v''}}\delta v'-\colorboxed{red}{\frac{d}{dx}\left(\frac{\partial L}{\partial v''}\right)}\delta v\right]_0^l+\int_0^l\colorboxed{green}{\frac{d^2}{dx^2}\left(\frac{\partial L}{\partial v''}\right)}\delta v dx\\
(D)\quad&\int_{t_1}^{t_2}\frac{\partial L}{\partial \dot{v}}\delta\dot{v} dt=\underbrace{\left[\frac{\partial L}{\partial\dot{v}}\delta v\right]_{t_1}^{t_2}}_{\textcolor{violet}{=0}}-\int_{t_1}^{t_2}\colorboxed{green}{\frac{\partial }{\partial t}\left(\frac{\partial L}{\partial \dot{v}}\right)}\delta v d t\\
(E)\quad&\int_{t_1}^{t_2}\int_0^l\frac{\partial L}{\partial \dot{v}'}\delta \dot{v}'dx dt = \underbrace{\left[\int_0^L\frac{\partial L}{\partial\dot{v}'}\delta v' dx\right]_{t_1}^{t_2}}_{\textcolor{violet}{\tiny \delta v(t_1)=\delta v(t_2) = 0}}-\int_{t_1}^{t_2}\int_0^l\frac{\partial}{\partial t}\left(\frac{\partial L}{\partial \dot{v}'}\right)\delta v' dx dt \\
&=\left[\left[\frac{\partial L}{\partial \dot{v}'}\delta v\right]_{t_1}^{t_2}\right]_0^l-\left[\int_0^l\frac{\partial}{\partial x}\left(\frac{\partial L}{\partial \dot{v}'}\right)\delta v dx\right]_{t_1}^{t_2}-\int_{t_1}^{t_2}\left[\colorboxed{red}{\frac{\partial}{\partial t}\left(\frac{\partial L}{\partial\dot{v}'}\right)}\delta v\right]_0^ldt\\
&+\int_{t_1}^{t_2}\int_0^l\colorboxed{green}{\frac{\partial}{\partial x}\left(\frac{\partial}{\partial t}\left(\frac{\partial L}{\partial\dot{v}'}\right)\right)}\delta v dx dt
\end{align*}

\important{$\int_{t_1}^{t_2}\left(\int_0^l\ \colorboxed{green}{\phantom{\sum}}\ \delta v dx+\left[\ \colorboxed{blue}{\phantom{\sum}}\ \delta v'\right]_0^l+\left[\ \colorboxed{red}{\phantom{\sum}}\ \delta v\right]_0^l\right)dt=0$}

\begin{align*}
&\text{Governing PDE:}\\
\colorboxed{green}{\phantom{|}}\quad& \frac{\partial L}{\partial v}-\frac{\partial}{\partial x}\left(\frac{\partial L}{\partial v'}\right)+\frac{\partial^2}{\partial x^2}\left(\frac{\partial L}{\partial v''}\right)-\frac{\partial }{\partial t}\left(\frac{\partial L}{\partial \dot{v}}\right)+\frac{\partial }{\partial x}\left(\frac{\partial }{\partial t}\left(\frac{\partial L}{\partial\dot{v}'}\right)\right)=0,\quad x\in[0;l]\\
&\text{Boundary conditions:}\\
\colorboxed{red}{\phantom{|}}\quad& \left[\frac{\partial L}{\partial v'}-\frac{\partial}{\partial x}\left(\frac{\partial L}{\partial v''}\right)-\frac{\partial}{\partial t}\left(\frac{\partial L}{\partial \dot{v}'}\right)\right]\delta v=0,\quad x=0,l\\
\colorboxed{blue}{\phantom{|} }\quad& \frac{\partial L}{\partial v''}\delta v'=0,\quad x=0,l
\end{align*}

\section{Rotations and angular velocity}

\begin{define}
A \textbf{rotation} $\tvec{R}$ is a linear operator that preserves length and respective orientation of vectors.
\end{define}

\importname{Length preservation}{$|\vec{r}|=|\tvec{R}\vec{r}|\ \forall\vec{r}\in\mathbb{R}^3$}

\mportant{$\tvec{R}$ is orthogonal}

\begin{align*}
|\vec{r}|^2=\vec{r}\cdot\vec{r}=\left(\tvec{R}\vec{r}\right)\cdot\left(\tvec{R}\vec{r}\right)=\left(\tvec{R}\vec{r}\right)^T\left(\tvec{R}\vec{r}\right)=\vec{r}^T\tvec{R}^T\tvec{R}\vec{r}=\vec{r}\cdot\left(\tvec{R}^T\tvec{R}\vec{r}\right)\\
\Longrightarrow\tvec{R}^T\tvec{R}=\tvec{I}\Longrightarrow\tvec{R}^T=\tvec{R}^{-1}
\end{align*}

\mportant{$\tvec{R}$ preserves orthogonality}

\begin{align*}
\vec{r}\bot\vec{s},\ \forall \vec{r},\vec{s}\in\mathbb{R}^3-\vec{0}\rightarrow\vec{r}\cdot\vec{s}=0=\left(\tvec{R}\vec{r}\right)\cdot\left(\tvec{R}\vec{s}\right)=\vec{r}^T\tvec{R}^T\tvec{R}\vec{s}=\vec{r}\cdot\vec{s}
\end{align*}

\mportant{$\tvec{R}$ preserves orientation}

Imagine a cube described by $\left(\vec{e}_x,\vec{e}_y,\vec{e}_z\right)$ with volume $V=1$ since:

\mportant{$\det\left(\left[\vec{e}_x;\vec{e}_y;\vec{e}_z\right]\right)=\vec{e}_z\cdot\left(\vec{e}_x\times\vec{e}_y\right)=1$}

Including a rotation:

\mportant{$\det\left(\left[\tvec{R}\vec{e}_x;\tvec{R}\vec{e}_y;\tvec{R}\vec{e}_z\right]\right)=\det\left(\tvec{R}\right)\det\left(\left[\vec{e}_x;\vec{e}_y\vec{e}_z\right]\right)=\det\left(\tvec{R}\right)=1$}

\important{$\det\left(\tvec{R}\right)=1$}

\begin{define}
Linear operators satisfying length preservation and orientation preservation belong to the \textbf{Special Orthogonality Group in 3D SO(3).}
\end{define}

Reflections do not belong to SO(3) but only to O(3) since the orientation preservation is not satisfied.

\subsection{Eigenvalues and eigenvectors of a rotation matrix}

\mportant{$\tvec{R}\vec{v}_i=\lambda_i\vec{v}_i,\ i=1,2,3\quad\Rightarrow\quad \tvec{R}\tvec{V}=\tvec{\Lambda}\tvec{V}$}

where $\tvec{V}=\left[\vec{v}_1;\vec{v}_2;\vec{v}_3\right]$ and $\tvec{\Lambda}=\diag\left(\lambda_1;\lambda_2;\lambda_3\right)$

\mportant{$\tvec{\Lambda}=\tvec{V}^{-1}\tvec{R}\tvec{V}$}

\begin{align*}
\det\left(\tvec{\Lambda}\right)=\lambda_1\lambda_2\lambda_3=\det\left(\tvec{V}^{-1}\right)\det\left(\tvec{R}\right)\det\left(\tvec{V}\right)=\det\left(\tvec{V}^{-1}\right)\det\left(\tvec{V}^{-1}\right)=\det\left(\tvec{V}^{-1}\tvec{V}\right)=1
\end{align*}

\important{$\det\left(\tvec{R}\right)=\lambda_1\lambda_2\lambda_3=1$}

$(\phantom{x})^\ast$ signifies complex conjugation and transpose.

\mportant{$\vec{v}^\ast\tvec{R}^T=\bar{\lambda}_i\vec{v}_i^\ast$}

\begin{align*}
\vec{v}_i^\ast\tvec{R}^T\tvec{R}\vec{v}_i=\vec{v}_i^\ast\vec{v}_i=\bar{\lambda}_i\lambda_i\vec{v}_i^\ast\vec{v}_i=|\lambda_i|^2\vec{v}_i^\ast\vec{v}_i
\end{align*}

\important{$|\lambda_1|=|\lambda_2|=|\lambda_3|=1$}

\important{$\lambda_1\lambda_2\lambda_3=1$}

A general solution of the above is

\important{$\lambda_1=1;\ \lambda_2=e^{i\Phi};\ \lambda_3=e^{-i\Phi}$}

The eigenvector associated to $\lambda_1=1$ is $\tvec{R}\vec{v}=\vec{v}$ which is the \textbf{axis of rotation}.

\subsection{Angle-axis representation}

\mportant{$\tvec{R}\vec{n}=\vec{n}$}

\begin{define}
The \textbf{Hermitian} of a matrix $\tvec{V}$ is written as $\tvec{V}^\ast$ which writes out as:

\end{define}

\mportant{$\tvec{V}=\begin{bmatrix}
\vec{n}&\vec{u}+i\vec{w}&\vec{u}-i\vec{w}
\end{bmatrix}
\qquad\tvec{V}^\ast=
\begin{bmatrix}
\vec{n}^T\\\left(\vec{u}-i\vec{w}^T\right)^T\\\left(\vec{u}-i\vec{w}\right)^T
\end{bmatrix}$}

\begin{align*}
\tvec{V}^\ast\tvec{V}=\begin{bmatrix}
\vec{n}^T\vec{n}&\vec{n}^T\left(\vec{u}+i\vec{w}\right)&\vec{n}^T\left(\vec{u}-i\vec{w}\right)\\
\left(\vec{u}-i\vec{w}\right)^T\vec{n}&\left(\vec{u}-i\vec{w}\right)^T\left(\vec{u}+i\vec{w}\right)&\left(\vec{u}-i\vec{w}\right)^T\left(\vec{u}-i\vec{w}\right)\\
\left(\vec{u}+i\vec{w}\right)^T\vec{n}&\left(\vec{u}+i\vec{w}\right)^T\left(\vec{u}+i\vec{w}\right)&\left(\vec{u}+i\vec{w}\right)^T\left(\vec{u}-i\vec{w}\right)
\end{bmatrix}=\tvec{I}
\end{align*}

which requires

\begin{align*}
\vec{n}^T\vec{u}&=\vec{n}^T\vec{w}=0\\
\vec{u}^T\vec{w}=0\\
\vec{u}^T\vec{u}+\vec{w}^T\vec{w}&=1\\
\vec{u}^T\vec{u}=\vec{w}^T\vec{w}
\end{align*}

Introducing the eigenvectors in the eigenvalue problem:

\begin{align*}
\tvec{R}\left(\vec{u}+i\vec{w}\right)&=e^{i\Phi}\left(\vec{u}+i\vec{w}\right)\\
\tvec{R}\left(\vec{u}+i\vec{w}\right)&=e^{i\Phi}\left(\vec{u}+i\vec{w}\right)
\end{align*}

Using Euler's formula and by summing and subtracting:

\begin{align*}
\tvec{R}\vec{u}&=\left(\vec{u}\cos\Phi-\vec{w}\sin\Phi\right)\\
\tvec{R}\vec{w}&=\left(\vec{u}\cos\Phi+\vec{w}\sin\Phi\right)
\end{align*}

\myspic{0.45}{Pictures/RotationVect}

The two eigenvectors of the rotation matrix span the plane perpendicular to the rotation axis $\vec{n}$. An arbitrary vector $\vec{x}$ can be written as:

\important{$\vec{x}=\underbrace{\left(\vec{x}\cdot\vec{n}\right)\vec{n}}_{\vec{x}_\parallel}-\underbrace{\vec{n}\times\left(\vec{n}\times\vec{x}\right)}_{\vec{x}_bot}$}

The rotated vector can then be expressed by the \textbf{Rodriguez formulae.}

\begin{align*}
\vec{y}=&\tvec{R}\vec{x}=\left(\vec{x}\cdot\vec{n}\right)\vec{n}-\vec{n}\times\left(\vec{n}\times\vec{x}\right)\cos\Phi+\left(\vec{n}\times\vec{x}\right)\sin\Phi\\
\vec{y}&=\tvec{R}\vec{x}=\vec{x}+\left(\vec{n}\times\vec{x}\right)\sin\Phi+(1-\cos\Phi)\vec{n}\times\left(\vec{n}\times\vec{x}\right)
\end{align*}

\important{$\tvec{N}=\begin{bmatrix}
0&-n_3&n_2\\n_3&0&-n_1\\-n_2&n_1&0
\end{bmatrix}$}

\mportant{$\tvec{N}\vec{x}=\vec{n}\times\vec{x}$}

\begin{align*}
\vec{y}&=\vec{x}+\sin\Phi\tvec{N}\vec{x}+(1-\cos\Phi)\tvec{N}^2\vec{x}=\tvec{R}\vec{x}\\
\tvec{R}&=\tvec{I}+\sin\Phi\tvec{N}+\left(1-\cos\Phi\right)\tvec{N}^2
\end{align*}

\subsection{Exponential map}

Consider a rotation: $\vec{r}=\tvec{R}\vec{r}_0$

\begin{align*}
\frac{\partial\vec{r}}{\partial \Phi}&=\frac{\partial\tvec{R}}{\partial \Phi}\vec{r}_0=\frac{\partial\tvec{R}}{\partial\Phi}\tvec{R}^T\vec{r}\\
&\text{where}\\
\frac{\partial\tvec{R}}{\partial\Phi}&=-\cos\Phi\tvec{N}+\sin\Phi\tvec{N}^2\\
&\text{which leads to}\\
\frac{\partial\tvec{R}}{\partial\Phi}\tvec{R}^T&=\left(-\cos\Phi\tvec{N}+\sin\Phi\tvec{N}^2\right)\left[\tvec{I}+\sin\Phi\tvec{N}+\left(1-\cos\Phi\tvec{N}^2\right)\right]^T
\end{align*}

Note that the following hold:

\begin{align*}
\tvec{N}^T=-\tvec{N}&\tvec{N}^3=-\tvec{N}\\
\tvec{N}^4=\tvec{N}\tvec{N}^3&=\tvec{N}(-\tvec{N})=-\tvec{N}^2
\end{align*}

Therefore:

\begin{align*}
\frac{\partial\tvec{R}}{\partial\Phi}\tvec{R}^T=\tvec{N}\\
\text{with boundary condition: } \vec{r}(\Phi = 0)=\vec{r}_0\\
\frac{\partial\vec{r}}{\partial\Phi}=\tvec{N}\vec{r}
\end{align*}

which, as a solution of the differential equation posed, admits the solution:

\important{$\vec{r}=e^{\tvec{N}\Phi}\vec{r}_0$}

\subsection{Small rotations}

\importname{\\Rotations are \textbf{NOT} commutative}{$\tvec{R}_1\tvec{R}_2=e^{\tvec{N}_1\Phi_1}e^{\tvec{N}_2\Phi_2}\neq e^{(\tvec{N}_1\Phi_1+\tvec{N}_2\Phi_2)}$}

\begin{align*}
\vec{y}&=\tvec{R}_1\vec{v}=\vec{v}+(\vec{n}_1\times\vec{v})\sin\Phi_1+(1-\cos\Phi_1(\vec{n}_1\times(\vec{n}_1\times\vec{v}))\\
&\approx\vec{v}+(\vec{n}_1\times\vec{v})d\Phi_1+(1-1)\vec{n}_1\times(\vec{n}_1\times\vec{v})=\vec{v}+(\vec{n}_1\times\vec{v})d\Phi_1
\end{align*}

When applying another small rotation it can be seen that the order can be reversed:

\begin{align*}
\vec{y}_2&=\vec{v}+(\vec{n}_1\times\vec{v})d\Phi_1+\vec{n}_2\times\left[\vec{v}+(\vec{n}_1\times\vec{v})d\Phi_1\right]d\theta_2\\
&=\vec{v}+(\vec{n}_1\times\vec{v})d\Phi_1+(\vec{n}_2\times\vec{v})d\Phi_2+\mathcal{O}(d\Phi_1^2,d\Phi_2^2,d\Phi_1d\Phi_2)
\end{align*}

\subsection{Cartesian Representation of Rotations}

\myspic{0.5}{CartesianRepresentation}

\mportant{$\vec{v}=v_x^\MA\vec{e}_x^\MA+v_y^\MA\vec{e}_y^\MA+v_z^\MA\vec{e}_z^\MA$}

After rotation we have:

\mportant{$\tvec{R}\vec{v}=v_x^\MA\tvec{R}\vec{e}_x^\MA+v_y^\MA\tvec{R}\vec{e}_y^\MA+v_z^\MA\tvec{R}e_z^\MA$}

Let's consider a B-frame that follows the rotation:

\begin{align*}
\tvec{R}\vec{v}&=v_x^\MB\vec{e}_x^\MB+v_y^\MB\vec{e}_y^\MB+v_z^\MB\vec{e}_z^\MB\\
&=v_x^\MA\vec{e}_x^\MB+v_y^\MA\vec{e}_y^\MB+v_z^\MA\vec{e}_z^\MB
\end{align*}

The components of the rotated vector in the A-frame are:

\begin{align*}
w_x^\MA&=\tvec{R}\vec{v}\cdot\vec{e}_x^\MA&=\left(v_x^\MA\vec{e}_x^\MB+v_y^\MA\vec{e}_y^\MB+v_z^\MA\vec{e}_z^\MB\right)\cdot\vec{e}_x^\MA\\
w_y^\MA&=\tvec{R}\vec{v}\cdot\vec{e}_y^\MA&=\left(v_x^\MA\vec{e}_x^\MB+v_y^\MA\vec{e}_y^\MB+v_z^\MA\vec{e}_z^\MB\right)\cdot\vec{e}_y^\MA\\
w_z^\MA&=\tvec{R}\vec{v}\cdot\vec{e}_z^\MA&=\left(v_x^\MA\vec{e}_x^\MB+v_y^\MA\vec{e}_y^\MB+v_z^\MA\vec{e}_z^\MB\right)\cdot\vec{e}_z^\MA
\end{align*}

which writes in matrix form as:

\importname{\\Active rotation matrix}{$\begin{bmatrix}
\basesca{w}{x}{A}\\\basesca{w}{y}{A}\\\basesca{w}{z}{A}
\end{bmatrix}=\begin{bmatrix}
\basevec{e}{x}{B}\cdot\basevec{e}{x}{A}&\basevec{e}{y}{B}\cdot\basevec{e}{x}{A}&\basevec{e}{z}{B}\cdot\basevec{e}{x}{A}\\\basevec{e}{x}{B}\cdot\basevec{e}{y}{A}&\basevec{e}{y}{B}\cdot\basevec{e}{y}{A}&\basevec{e}{z}{B}\cdot\basevec{e}{y}{A}\\\basevec{e}{x}{B}\cdot\basevec{e}{z}{A}&\basevec{e}{y}{B}\cdot\basevec{e}{z}{A}&\basevec{e}{z}{B}\cdot\basevec{e}{z}{A}
\end{bmatrix}
\begin{bmatrix}
\basesca{v}{x}{A}\\\basesca{v}{y}{A}\\\basesca{v}{z}{A}
\end{bmatrix}$}

Consider now a frame rotation:

\begin{align*}
\vec{v}&=\basesca{v}{x}{B}\basevec{e}{x}{B}+\basesca{v}{y}{B}\basevec{e}{y}{B}+\basesca{v}{z}{B}\basevec{e}{z}{B}\\
&=\basesca{v}{x}{A}\basevec{e}{x}{A}+\basesca{v}{y}{A}\basevec{e}{y}{A}+\basesca{v}{z}{A}\basevec{e}{z}{A}
\end{align*}

The components of $\vec{v}$ in the B-frame are given by:

\begin{align*}
\basesca{v}{x}{B}&=\vec{v}\cdot\basevec{e}{x}{B}&=\left(\basesca{v}{x}{A}\basevec{e}{x}{A}+\basesca{v}{y}{A}\basevec{e}{y}{A}+\basesca{v}{z}{A}\basevec{e}{z}{A}\right)\cdot\basevec{e}{x}{B}\\
\basesca{v}{y}{B}&=\vec{v}\cdot\basevec{e}{y}{B}&=\left(\basesca{v}{x}{A}\basevec{e}{x}{A}+\basesca{v}{y}{A}\basevec{e}{y}{A}+\basesca{v}{z}{A}\basevec{e}{z}{A}\right)\cdot\basevec{e}{y}{B}\\
\basesca{v}{z}{B}&=\vec{v}\cdot\basevec{e}{z}{B}&=\left(\basesca{v}{x}{A}\basevec{e}{x}{A}+\basesca{v}{y}{A}\basevec{e}{y}{A}+\basesca{v}{z}{A}\basevec{e}{z}{A}\right)\cdot\basevec{e}{z}{B}
\end{align*}

written in matrix form:

\importname{\\Passive rotation matrix}{$\begin{bmatrix}
\basesca{v}{x}{B}\\
\basesca{v}{y}{B}\\
\basesca{v}{z}{B}
\end{bmatrix}
=
\underbrace{\begin{bmatrix}
\basevec{e}{x}{A}\cdot\basevec{e}{x}{B}&
\basevec{e}{y}{A}\cdot\basevec{e}{x}{B}&
\basevec{e}{z}{A}\cdot\basevec{e}{x}{B}\\
\basevec{e}{x}{A}\cdot\basevec{e}{y}{B}&
\basevec{e}{y}{A}\cdot\basevec{e}{y}{B}&
\basevec{e}{z}{A}\cdot\basevec{e}{y}{B}\\
\basevec{e}{x}{A}\cdot\basevec{e}{z}{B}&
\basevec{e}{y}{A}\cdot\basevec{e}{z}{B}&
\basevec{e}{z}{A}\cdot\basevec{e}{z}{B}\\
\end{bmatrix}}_ {\rotmat{B}{R}{A}}
\begin{bmatrix}
\basesca{v}{x}{A}\\
\basesca{v}{y}{A}\\
\basesca{v}{z}{A}
\end{bmatrix}
$}

\subsection{Euler Angles}

\myspic{0.5}{EulerAngles}

\importname{\\Precession}{$\begin{bmatrix}
x'\\y'\\z'
\end{bmatrix}
=\tvec{R}_\Psi\begin{bmatrix}
X\\Y\\Z
\end{bmatrix}
=\begin{bmatrix}
\cos\Psi&-\sin\Psi&0\\
\sin\Psi&\cos\Psi&0\\
0&0&1
\end{bmatrix}
\begin{bmatrix}
X\\Y\\Z
\end{bmatrix}
$}

\importname{\\Nutation}{$
\begin{bmatrix}
x''\\y''\\z''
\end{bmatrix}
=
\tvec{R}_\theta\begin{bmatrix}
x'\\y'\\z'
\end{bmatrix}
=
\begin{bmatrix}
\cos\theta&0&-\sin\theta\\
0&1&0\\
\sin\theta&0&\cos\theta
\end{bmatrix}
\begin{bmatrix}
x'\\y'\\z'
\end{bmatrix}
$}

\importname{\\Spin}{$
\begin{bmatrix}
x\\y\\z
\end{bmatrix}
=\tvec{R}_\Phi\begin{bmatrix}
x'\\y'\\z'
\end{bmatrix}
=
\begin{bmatrix}
\cos\theta&-\sin\theta&0\\
\sin\theta&\cos\theta&0\\
0&0&1
\end{bmatrix}
\begin{bmatrix}
x''\\y''\\z''
\end{bmatrix}
$}

The resulting rotation is a incommutable sequence of three rotations.

\mportant{$\begin{bmatrix}
x\\y\\z
\end{bmatrix}=
\underbrace{\tvec{R}_\Phi\tvec{R}_\theta\tvec{R}_\Psi}_{\tvec{R}}\begin{bmatrix}
X\\Y\\Z
\end{bmatrix}$}

\important{$
\tvec{R}=\begin{bmatrix}
\cos\Phi\cos\theta\cos\Psi-\sin\Phi\sin\Psi&
\cos\Phi\cos\theta\sin\Psi+\sin\Phi\cos\Psi&
-\cos\Phi\cos\theta\\
\sin\Phi\cos\theta\cos\Psi-\cos\Phi\sin\Psi&
-\sin\Phi\cos\theta\sin\Psi+\cos\Phi\cos\Psi&
\sin\Phi\cos\theta\\
\sin\theta\cos\Psi&\sin\theta\sin\Psi&\cos\theta
\end{bmatrix}
$}

\begin{small}
\begin{align*}
\tvec{R}(\Psi,\theta,\Phi)&=\tvec{R}(\Psi+\pi,-\theta,\Phi+\pi)&=\tvec{R}(\Psi+\pi,-\theta,\Phi-\pi)\\
&=\tvec{R}(\Psi-\pi,-\theta,\Phi+\pi)=\tvec{R}(\Psi-\pi,-\theta,\Phi-\pi)
\end{align*}
\end{small}

Therefore the range of angles is restricted to:

\important{$0\leq\Psi<2\pi;\ 0\leq\theta\leq\pi;\ 0\leq\Psi<2\pi$}

\subsection{Angular velocity}

Consider a time dependent rotation, smooth in t

\mportant{$\vec{r}(t)=\tvec{R}(t)\vec{r}(0)\quad \tvec{R}(t)\in SO(3)$}

The velocity is given by:

\begin{align*}
\vecd{r}(t)&=\tvec{\dot{R}}(t)\vec{r}(0)=\tvec{\dot{R}}\tvec{R}^{-1}\vec{r}(t)=\tvec{\dot{R}}\tvec{R}^T\vec{r}(t)\\
\tvec{R}\tvec{R}^T&=\tvec{I}\Rightarrow\tvec{\dot{R}}\tvec{R}^T+\tvec{R}\tvec{\dot{R}}^T=\tvec{0};\quad \tvec{\dot{R}}\tvec{R}^T=-\tvec{R}\tvec{\dot{R}}^T=-(\tvec{\dot{R}}\tvec{R}^T)^T
\end{align*}

\begin{define}
Define $t\vec{\Omega}=\tvec{\dot{R}}\tvec{R}^T$ as the \textbf{angular velocity}. It is a skew symmetric matrix: $\tvec{\Omega}=-\tvec{\Omega}^T$
\end{define}

The velocity becomes:

\important{$\vecd{r}(t)=\tvec{\Omega}(t)\vec{r}(t)$}

The structure of the velocity matrix allows to write:

\mportant{$\tvec{\Omega}=-\tvec{\Omega}^T\Rightarrow\tvec{\Omega}\vec{r}=\begin{bmatrix}
0&\omega_{12}&\omega_{13}\\-\omega_{12}&0&\omega_{23}\\-\omega_{13}&-\omega_{23}&0
\end{bmatrix}
\begin{bmatrix}
r_1\\r_2\\r_3
\end{bmatrix}
=
\begin{bmatrix}
\omega_{12}r_2-\omega_{13}r_3\\
-\omega_{12}r_1+\omega_{23}r_3\\
-\omega_{13}r_1-\omega_{23}r_2
\end{bmatrix}
=
\begin{bmatrix}
\vec{e}_1&\vec{e}_2&\vec{e}_3\\
-\omega_{23}&\omega_{13}&\omega_{12}\\
r_1&r_2&r_3
\end{bmatrix}
=
\vec{\omega}\times\vec{r}$}

\important{$\vecd{r}(t)=\vec{\omega}(t)\times\vec{r}(t)\quad\text{where}\quad\vec{\omega}=\begin{bmatrix}
-\omega_{23}\\\omega_{13}\\-\omega_{12}
\end{bmatrix}$}

\subsubsection{Uniqueness of the angular velocity}

\begin{align*}
\vec{v}_C&=\vec{v}_A+\vec{\omega}_A\times\vec{r}_{AC}\\
\vec{v}_C&=\vec{v}_B+\vec{\omega}_B\times\vec{r}_{BC}
\vec{v}_B-\vec{v}_A=\vec{\omega}_A\times\vec{r}_{AB}=-\vec{\omega}_B\times\vec{r}_{BC}+\vec{\omega}_A\times\vec{r}_{AC}\\
&\Rightarrow\vec{\omega}_A\times(\vec{r}_{AB}-\vec{r}_{AC})=-\vec{\omega}_B\times\vec{r}_{BC}
(\vec{\omega}_A-\vec{\omega}_B)\times\vec{r}_{BC}=\vec{0}\\
&\Longrightarrow\vec{\omega}_A=\vec{\omega}_B,\ \forall\ A,B\in\mathcal{B}
\end{align*}

\subsubsection{Finding the angular velocity}

\begin{enumerate}
\item Use the velocity transfer formula twice:

\begin{align*}
\vec{v}_B-\vec{v}_A&=\vec{\omega}\times\vec{r}_{AB}\\
\vec{v}_C-\vec{v}_A&=\vec{\omega}\times\vec{r}_{AC}
\end{align*}

\item The two planes spanned by $(\vec{r}_{AB},\vec{omega})$ and $\vec{r}_{AC},\vec{\omega})$ define $\vec{e}_\omega$ with their intersection.

\item The magnitude of omega can be found by inserting $\omega\cdot\vec{e}_{\omega}$ into one of the equations.
\end{enumerate}

Special case: $\vec{v}_B=\vec{v}_A\Rightarrow\vec{\omega}\times\vec{r}_{AB}=0\Rightarrow\vec{\omega}\parallel\vec{r}_{AB}$

This relation defines the instantaneous axis of rotation.

\subsubsection{Additivity of Rotational Velocities}

Rotations are not additive: $(\tvec{A}\tvec{B}\neq\tvec{B}\tvec{A})\forall\tvec{A},\tvec{B}\in SO(3)$

\begin{align*}
\vec{r}_{AB}(t)=\tvec{R}_n(t)\tvec{R}_{n-1}(t)\ldots\tvec{R}_1(t)\vec{r}_{AB}(0)
\end{align*}

Then the velocity writes as:

\begin{align*}
\vecd{r}_{AB}(t)=\left(\tvec{\dot{R}}_n(t)\tvec{R}_{n-1}(t)\ldots\tvec{R}_1(t)+\tvec{R}_n(t)\tvec{\dot{R}}_{n-1}(t)\ldots\tvec{R}_1(t)+\tvec{R}_n\tvec{R}_{n-1}\ldots\tvec{\dot{R}}_1(t)\right)\vec{r}_{AB}(0)
\end{align*}

The time derivative of a rotation evaluates as:

\mportant{$\tvec{R}\tvec{R}^T=\tvec{I}\Rightarrow\tvec{\dot{R}}\tvec{R}^T+\tvec{R}\tvec{\dot{R}}^T=\tvec{0}\quad\tvec{\dot{R}}\tvec{R}^T\tvec{R}=-\tvec{R}\tvec{\dot{R}}^T\tvec{R}\quad\tvec{\dot{R}}=-\tvec{R}\tvec{\dot{R}}^T\tvec{R}$}

Substitution yields:

\mportant{\small$\vecd{r}_{AB}=\left(-\tvec{R}_n\tvec{\dot{R}}_n^T\tvec{R}_n\tvec{R}_{n-1}\ldots\tvec{R}_1+\tvec{R}_n(-\tvec{R}_{n-1}\tvec{\dot{R}}_{n-1}^T\tvec{R}_{n-1})\tvec{R}_{n-2}\ldots\tvec{R}_1+\tvec{R}_n\ldots(-\tvec{R}_1\tvec{\dot{R}}_1^T\tvec{R}_1)\right)\vec{r}_{AB}(0)$\normalsize}

\mportant{If $t\rightarrow 0\quad\tvec{R}_i\rightarrow\tvec{I}\quad-\tvec{R}_i\tvec{\dot{R}}_i^T=\tvec{\Omega}_i\Rightarrow$}

\mportant{$\vecd{r}_{AB}=(\tvec{\Omega}_n+\tvec{\Omega}_{n-1}+\ldots+\tvec{\Omega}_1)\vec{r}_{AB}(0)=\sum\limits_{i=1}^n\vec{\omega}_i\times\vec{r}_{AB}(0)$}

But we also found: $\vecd{r}_{AB}=\vec{v}_{AB}=\vec{\omega}\times\vec{r}_{AB}(0)\Rightarrow$

\important{$\vec{\omega}=\sum\limits_{i=1}^n\vec{\omega}_i$}

\subsection{Jacobian of Rotation}

\myspic{0.5}{EulerAnglesJacobian}

The rotational velocity based on euler angles is calculated as follows:

\mportant{$\vec{\omega}=\dot{\Psi}\vec{K}+\dot{\theta}\vec{j}'+\dot{\Phi}\vec{k}$}

We need a common set of components: $\begin{bmatrix}
\vec{i}&\vec{j}&\vec{k}
\end{bmatrix}$

Use rotations to express $\vec{K}$ and $\vec{j}'$ in the $\begin{bmatrix}
\vec{i}&\vec{j}&\vec{k}
\end{bmatrix}$ frame.

\mportant{$\vec{k}=\tvec{R}_\Phi\tvec{R}_\theta\tvec{R}_\Psi\vec{K}=\tvec{R}_\Phi\tvec{R}_\theta\begin{bmatrix}
0\\0\\1
\end{bmatrix}=\tvec{R}_\Phi\begin{bmatrix}
-\sin\theta\\0\\\cos\theta
\end{bmatrix}=\begin{bmatrix}
-\cos\Phi\sin\theta\\
\sin\Phi\sin\theta\\
\cos\theta
\end{bmatrix}$}

\mportant{$\vec{j}=\tvec{R}_\Phi\tvec{R}_\theta\vec{j}'=\begin{bmatrix}
\sin\Phi\\
\cos\Phi\\
0
\end{bmatrix}$}

Thus the angular velocity expressed in the body frame is:

\mportant{$\vec{\omega}=\begin{bmatrix}
-\dot{\Psi}\sin\theta\cos\Phi+\dot{\theta}\sin\Phi\\
-\dot{\Psi}\sin\theta\sin\Phi+\dot{\theta}\cos\Phi\\
-\dot{\Phi}+\dot{\Psi}\cos\theta
\end{bmatrix}$}

which can be written as:

\important{$\vec{\omega}^B=\begin{bmatrix}
-\sin\theta\cos\Phi&\sin\Phi&0\\
-\sin\theta\sin\Phi&\cos\Phi&0\\
\cos\theta&0&-1
\end{bmatrix}\begin{bmatrix}
\dot{\Psi}\\
\dot{\theta}\\
\dot{\Phi}
\end{bmatrix}
=\tvec{J}^B(\vec{q})\vecd{q}$}

\begin{itemize}
\item Having the angular velocity of the body allows calculating the change of the angles, by inverting the Jacobian (only possible for certain configurations) and integrating the resulting angle-velocities.
\end{itemize}

Likewise the angular velocity can be expressed in fixed axes:

\mportant{$\vec{\omega}=\dot{\Psi}\vec{K}+\dot{\theta}\vec{j}'+\dot{\Phi}\vec{k}$}

The intermediate axes are projected onto the initial axes:

\mportant{$\vec{k}=\tvec{R}_\Phi\tvec{R}_\theta\tvec{R}_\Psi\vec{K} \Rightarrow \vec{K}=\tvec{R}_\Psi^T\tvec{R}_\theta^T\tvec{R}_\Phi^T\vec{k}\qquad\vec{j}'=\tvec{R}_\Psi\vec{J}\Rightarrow\tvec{R}_\Psi^T\vec{j}'$}

and therefore

\begin{align*}
\vec{\omega}&=\dot{\Psi}\begin{bmatrix}
0\\0\\1
\end{bmatrix}+\dot{\theta}\tvec{R}_\Psi^T\begin{bmatrix}
0\\1\\0
\end{bmatrix}+\dot{\Phi}\tvec{R}_\Psi^T\tvec{R}_\theta^T\tvec{R}_\Phi^T\begin{bmatrix}
0\\0\\1
\end{bmatrix}\\
&=\dot{\Psi}\begin{bmatrix}
0\\0\\1
\end{bmatrix}+\dot{\theta}\begin{bmatrix}
0\\\cos\theta\\\sin\theta
\end{bmatrix}+\dot{\Phi}\begin{bmatrix}
\sin\Phi\sin\theta\\
-\cos\Phi\sin\theta\\
\cos\theta
\end{bmatrix}
\end{align*}

which can be written as:

\important{$\vec{\omega}^I=\begin{bmatrix}
0&0&\sin\Phi\sin\theta\\
0&\cos\theta&-\cos\Phi\sin\theta\\
1&\sin\theta&\cos\theta
\end{bmatrix}
\begin{bmatrix}
\dot{\Psi}\\
\dot{\theta}\\
\dot{\Phi}
\end{bmatrix}
=\tvec{J}^I(\vec{q})\vecd{q}$}

\subsection{Derivative of a vector in a moving frame}

\mportant{$\vecd{u}=\sum\limits_{i=1}^3\dot{u}_i\vec{e}_i+\sum\limits_{i=1}^3 u_i\vecd{e}_i=\sum\limits_{i=1}^3 \dot{u}_i\vec{e}_i+\sum\limits_{i=1}^3 u_i\vec{\omega}\times\vec{e}_i$}

\important{$\vecd{u}=\vec{\overset{\circ}{u}}+\vec{\omega}\times\vec{u}$}

\begin{itemize}
\item $\vec{\overset{\circ}{u}}$ is the derivative of the considered quantity without knowing that we are in a rotating frame $\rightarrow$ calculate the derivative naïvely.
\end{itemize}

\subsection{Euler equations}

\begin{align*}
I_1\dot{\omega}_1+(I_3-I_2)\omega_2\omega_3&=M_1\\
I_2\dot{\omega}_2+(I_1-I_3)\omega_1\omega_3&=M_2\\
I_3\dot{\omega}_3+(I_2-I_1)\omega_1\omega_2&=M_3
\end{align*}
\mportant{Rotational Euler equations}

When using body components of $\vec{v}_C$, the LMP becomes:

\mportant{$\vecd{v}_C=\overset{\circ}{\vec{v}_C+\vec{\omega}\times\vec{v}_C}$}


\begin{align*}
m_{tot}\left[\dot{v}_1+v_3\omega_2-v_2\omega_3\right]&=F_1\\
m_{tot}\left[\dot{v}_2+v_1\omega_3-v_3\omega_1\right]&=F_2\\
m_{tot}\left[\dot{v}_3+v_2\omega_1-v_1\omega_2\right]&=F_3
\end{align*}
\mportant{Translational Euler Equations}

To find the orientation of the body, the relation between the angular velocity and the rates of the rotation parameters have to be found:

\mportant{$\vec{\omega}^B=\begin{bmatrix}
-\sin\theta\cos\Phi&\sin\Phi&0\\
-\sin\theta\sin\Phi&\cos\Phi&0\\
\cos\theta&0&-1
\end{bmatrix}\begin{bmatrix}
\dot{\Psi}\\\dot{\theta}\\\dot{\Phi}
\end{bmatrix}=\tvec{J}_\omega(\vec{q})\vecd{q}$}

The above needs to be inverted and integrated in time in order to find the rotation parameters. This can be difficult or even impossible. For example when $\theta =k\pi\ k=0,1,\ldots$ the relation becomes singular:

\mportant{$\vec{\omega}=\begin{bmatrix}
0&\sin\Phi&0\\
0&\cos\Phi&0\\
\pm1&0&-1
\end{bmatrix}\begin{bmatrix}
\dot{\Psi}\\
\dot{\theta}\\
\dot{\Phi}
\end{bmatrix}$}

Since in that case the relation can not be inverted. Therefore the Euler angle rates are not deducible form the angular velocity components. Possible remedies are:

\begin{itemize}
\item Choose another sequence of rotation such that the singularity is not occurring in the range of interest.
\item Use redundant (singularity free) formulation i.e. quaternions.
\end{itemize}

\section{Equations of Motion of a 3D rigid body using Lagrange Formalism}

\begin{align*}
\MT&=\onha m_{tot}|\vec{v}_C|^2+\onha\vec{\omega}^T\tvec{I}_C\vec{\omega}\\
\MT&=\onha m_{tot}|\vec{v}_C|^2+\onha\vecd{q}^T\tvec{J}^T(\vec{q})\tvec{I}_C\tvec{J}(\vec{q})\vecd{q}
\end{align*}

with generalized coordinates chosen as $\vec{q}=\begin{bmatrix}
x_C&y_C&z_C&\Psi&\theta&\Phi
\end{bmatrix}$

The virtual direction to compute the generalized forces can be obtained by the kinematic method applied to the Jacobians:

\begin{align*}
\omegavec&=\begin{bmatrix}
-\sin\theta\cos\Phi&\sin\Phi&0\\
-\sin\theta\sin\Phi&\cos\Phi&0\\
\cos\theta&0&-1
\end{bmatrix}\begin{bmatrix}
\Psid\\\thetad\\\Phid
\end{bmatrix}
=\tvec{J}_\omega(\vec{q})\vecd{q}\\
\vec{\delta\theta}&=\begin{bmatrix}
-\sin\theta\cos\Phi&\sin\Phi&0\\
-\sin\theta\sin\Phi&\cos\Phi&0\\
\cos\theta&0&-1
\end{bmatrix}
\begin{bmatrix}
\delta\Psi\\
\delta\theta\\
\delta\Phi
\end{bmatrix}
=
\tvec{J}_\omega(\vec{q})\delta\vec{q}
\end{align*}

\section{Linearization}
\subsection{Equilibrium and stability}
\subsubsection{Definition of equilibrium}
\begin{define}
An \textbf{equilibrium} is a time independent configuration in the generalized coordinates.
\end{define}

\important{$\vec{q}(t)=\vec{q}(0)=\vec{q}_{eq}\ \forall t\Longrightarrow\vecd{q}(t)=\vec{0}\ \vecdd{q}(t)=\vec{0}$}

Consider the Lagrange equations:

\mportant{$\ddt\left(\frac{\partial\MT}{\partial\vecd{q}}\right)=\frac{\partial\MT}{\partial\vec{q}}+\frac{\partial\MV}{\partial\vec{q}}-\vec{Q}^{nc}=\vec{0}$}

Which can be compactly written as 

\mportant{$\vec{f}(\vec{q},\vecd{q},\vecdd{q},t)=\vec{0}$}

For the equilibrium:

\important{$\vec{f}(\vec{q}=\vec{q}_{eq},\vecd{q}=\vec{0},\vecdd{q}=\vec{0})=\vec{0}$}

In general: $\vec{r}_k=\vec{r}_k(\vec{q},t)$ is the cartesian position of particle $k$. The velocity of that particle is:

\mportant{$\vecd{r}_k=\frac{\partial\vec{r}_k}{\partial\vec{q}}\vecd{q}+\frac{\partial\vec{r}_k}{\partial t}=\tvec{A}_k(\vec{q},\vecd{q},t)\vecd{q}+\vec{b}_k(\vec{q},\vecd{q},t)$}

Thus the kinetic energy is composed by three terms:

\important{\small$\MT=\underbrace{\sum\limits_{k=1}^n\onha m_k\left(\frac{\partial\vec{r}_k}{\partial t}\right)^T\left(\frac{\partial\vec{r}_k}{\partial t}\right)}_{T_0}+\underbrace{\onha\sum\limits_{k=1}^n m_k\left(\frac{\partial\vec{r}_k}{\partial t}\right)^T\left(\frac{\partial\vec{r}_k}{\partial \vec{q}}\right)\vecd{q}}_{T_1}+\underbrace{\onha\sum\limits_{k=1}^nm_k\vecd{q}^T\left(\frac{\partial\vec{r}_k}{\partial\vec{q}}\right)^T\left(\frac{\partial\vec{r}_k}{\partial\vec{q}}\right)\vecd{q}}_{T_2}$\normalsize}

$T_0$, the transport kinetic energy describes the KE only due to the imposed overall motion proper of rheonomic systems. $T_2$, the relative kinetic energy does not depend on the overall motion, but only on the motion relative to the imposed movement. $T_1$ is the coupling kinetic energy.

\subsubsection{Equilibrium for sceleronomic systems}

When no imposed motion is present the general velocity reduces to:

\mportant{$\vec{r}_k=\pfrac{\vec{r}_k}{\vec{q}}\vecd{q}=\tvec{A}_k(\vec{q},t)\vecd{q}$}

\begin{footnotesize}
Note that $\frac{\partial \vec{r}_k}{\partial \vec{q}}$ represents a mapping from the generalized coordinate space to the cartesian coordinate space.
\end{footnotesize}

\mportant{$\Longrightarrow \MT = T_2 = \onha\vecd{q}^T\tvec{M}(\vec{q},t)\vecd{q}\quad \tvec{M}(\vec{q}=\sum\left(\pfrac{\vec{r}_k}{\vecd{q}}\right)^T\left(\pfrac{\vec{r}_k}{\vecd{q}}\right)$}

Consider a conservative system:

\mportant{$\underbrace{\ddt\left(\pfrac{\MT}{\vecd{q}}\right)-\pfrac{\MT}{\vec{q}}_{\vec{f}_\MT}}+\underbrace{\pfrac{\MV}{\vec{q}}}_{\vec{f}_\MV}=\vec{0}\qquad \vec{f}_\MT+\vec{f}_\MV=\vec{0}$}

At equilibrium the contributions to the LE are individually zero since both contain terms in $\vecd{q},\vecdd{q}$ which are zero by definition.

\mportant{$\ddt\pfrac{\MT}{\vecd{Q}}-\pfrac{\MT}{\vec{q}}=0$}

\important{$\vec{f}_\MV=\pfrac{\MV}{\vec{q}}=\vec{0}$}

The equilibrium equation for holonomic, conservative systems. In general this is a nonlinear set of equations that admits multiple (even non-physical) solutions.

\subsubsection{Newton-Raphson method}

How to find $\vec{q}_{eq}$ ?

\mportant{$\vec{f}_\MV(\vec{q}_{eq}=\vec{0}$}

\begin{enumerate}
\item Start from a guess $\vec{q}_0$
\item Evaluate an equilibrium residual

\mportant{$\vec{f}_\MV(\vec{q}_0)=\vec{e}_0\neq\vec{0}$}
\item Correct initial guess

\mportant{$\vec{q}_{eq}=\vec{q}_0+\Delta\vec{q}$}
\item Expand equilibrium equation in Taylor series around $\vec{q}_0$:

\mportant{$\vec{f}_\MV(\vec{q}_0+\Delta\vec{q})=\vec{f}_\MV(\vec{q}_0)+\left.\pfrac{\vec{f}_\MV}{\vec{q}}\right|_{vec{q}_0}\Delta\vec{q}+\ldots=\vec{0}$}
\item Solve for correction

\mportant{$\left.\pfrac{\vec{f}_\MV}{\vec{q}}\right|_{\vec{q}_0}\Delta\vec{q}=-\vec{f}_\MV(\vec{q}_0)=\vec{e}_0$}

\item Judge convergence

\mportant{$\vec{f}_\MV(\vec{q}_0+\Delta\vec{q})=\vec{f}_\MV(\vec{q}_1)=\vec{e}_1$}

If $||\vec{e}_1||>Tolerance$, $\vec{q}_2=\vec{q}_1+\Delta\vec{q}$ go to 4. and repeat until convergence.
\end{enumerate}

If non conservative generalized forces are present:

\important{$\vec{f}_V(\vec{q}_{eq})-\vec{Q}^{nc}(\vec{q}_{eq},\vecd{q}=\vec{0})=\vec{0}$}

\subsubsection{Equilibrium for rheonomic systems}

\mportant{$\MT=T_0(\vec{q},t)+T_1(\vec{q},\vecd{q},t)+T_2(\vec{q},\vecd{q},t)$}

where $T_0$ is $0^{th}$ order in $\vecd{q}$, $T_1$ is linear in $\vecd{q}$ and $T_2$ is quadratic in $\vecd{q}$. All are arbitrary in $\vec{q}$.

\mportant{$\vec{f}_T=\ddt\pfrac{\MT}{\vecd{q}}-\pfrac{\MT}{\vec{q}}\qquad\vec{f}_\MV=\pfrac{\MV}{\vec{q}}\qquad\vec{Q}^{nc}=\vec{Q}^{nc}(\vec{q},\vecd{q})$}

\mportant{$\left.\vec{f}_\MT\right|_{eq}=\left.\left[\ddt\left(\pfrac{T_0}{\vecd{q}}+\pfrac{T_1}{\vecd{q}}+\pfrac{T_2}{\vecd{q}}\right)-\pfrac{T_0}{\vec{q}}-\pfrac{T_1}{\vec{q}}-\pfrac{T_2}{\vec{1}}\right]\right|_{eq}$}

The equilibrium equation for rheonomic systems writes:

\important{$\ddt\left.\pfrac{T_1}{\vecd{q}}\right|_{eq}-\left.\pfrac{T_0}{\vec{q}}\right|_{eq}+\left.\pfrac{\MV}{\vec{q}}\right|_{eq}-\left.\vec{Q}^{nc}\right|_{eq}=\vec{0}$}

For conservative systems define: $\MV^+=\MV - T_0$

\important{$\ddt\left.\pfrac{T_1}{\vecd{q}}\right|_{eq}+\left.\pfrac{\MV^+}{\vec{q}}\right|_{eq}=\vec{0}$}

The major difference with sceleronomic systems is that the imposed motion brings extra generalized forces.

\subsubsection{Linearized equations}

Assumption: $\vec{q}=\vec{q}_{eq}+\vec{\tilde{q}}$ $\vecd {q}=\vecd{\tilde{q}}$ $\vecdd{q}=\vecdd{\tilde{q}}$ with $\vec{\tilde{q}}$ \textbf{small.}

\mportant{$\vec{f}(\vec{q},\vecd{q},\vecdd{q})=\underbrace{\vec{f}(\vec{q}_{eq},\vec{0},\vec{0})}_{0 by def.}+\left.\pfrac{\vec{f}}{\vec{q}}\right|_{eq}\vec{\tilde{q}}+\left.\pfrac{\vec{f}}{\vecd{q}}\right|_{eq}\vecd{\tilde{q}}+\left.\pfrac{\vec{f}}{\vecdd{q}}\right|_{eq}\vecdd{\tilde{q}}+H.O.T.$}

\begin{align*}
\left.\pfrac{\vec{f}}{\vec{q}}\right|_{eq}&=\tvec{K}&\text{Stiffness matrix}\\
\left.\pfrac{\vec{f}}{\vecd{q}}\right|_{eq}&=\tvec{D}&\text{Damping matrix}\\
\left.\pfrac{\vec{f}}{\vecdd{q}}\right|_{eq}&=\tvec{M}&\text{Mass matrix}
\end{align*}

Thus the system can be compactly written as 

\important{$\tvec{M}\vecdd{\tilde{q}}+\tvec{D}\vecd{\tilde{q}}+\tvec{K}\vec{\tilde{q}}=\vec{0}$}

\subsection{Linearized equations for sceleronomic systems}

\mportant{$\vec{f}(\vec{q},\vecd{q},\vecdd{q})=\vec{f}_\MT(\vec{q},\vecd{q},\vecdd{Q})+\vec{f}_\MV(\vec{q})-\vec{Q}^{nc}(\vec{q},\vecd{q})=\vec{0}$}

For sceleronomic systems $\MT=T_2(\vecd{q},\vec{q})$ quadratic in $\vecd{q}$.

\mportant{\small$\tvec{M}=\left.\pfrac{\vec{f}_\MT}{\vecdd{q}}\right|_{eq}=\pfrac{}{\vecdd{q}}\left[\ddt\left(\pfrac{\MT_2}{\vecd{q}}\right)-\pfrac{\MT_2}{\vec{q}}\right]=\pfrac{}{\vecdd{q}}\left[\textcolor{red}{\pfrac{}{\vec{q}}\left(\pfrac{\MT_2}{\vecd{q}}\right)\vecd{q}}+\textcolor{green}{\pfrac{}{\vecd{q}}\left(\pfrac{\MT_2}{\vecd{q}}\right)\vecdd{q}}-\textcolor{blue}{\pfrac{\MT_2}{\vec{q}}}\right]$\normalsize}

\begin{align*}
\pfrac{}{\vecdd{q}}\textcolor{red}{\left[\pfrac{}{\vec{q}}\left(\pfrac{\MT_2}{\vecd{q}}\right)\vecd{q}\right]}&=\vec{0}\\
\pfrac{}{\vecdd{q}}\textcolor{green}{\left[\pfrac{}{\vec{q}}\left(\pfrac{\MT_2}{\vecd{q}}\right)\vecdd{q}\right]}&=\frac{\partial^2\MT_2}{\partial\vecd{q}\partial\vecd{q}}\\
\pfrac{}{\vecdd{q}}\textcolor{blue}{\left[-\pfrac{\MT_2}{\vec{q}}\right]}&=\vec{0}
\end{align*}

Thus the mass matrix can be described as:

\important{$\tvec{M}= \left.\frac{\partial^2\MT_2}{\partial\vecd{q}\partial\vecd{q}}\right|_{eq}$}

The mass matrix is symmetric:

\mportant{$M_{ij}=\frac{\partial^2\MT_2}{\partial\vecd{q}_i\partial\vecd{q}_j}=\frac{\partial^2\MT_2}{\partial\vecd{q}_j\partial\vecd{q}_i}=M_{ji}$}

Further the mass matrix is positive definite and the kinetic energy is a quadratic form of the mass matrix:

\mportant{$\frac{\partial^2\MT_2}{\partial\vecd{q}_i\partial\vecd{q}_j}=\tvec{M}\Longrightarrow\MT_2=\onha\vecd{q}^T\tvec{M}\vecd{q}>0\ \forall\vecd{q}$}

Stiffness matrix: Consider a conservative system $\vec{Q}^{nc}=\vec{0}$:

\mportant{$\left.\left[\pfrac{\vec{f}_\MT}{\vec{q}}+\pfrac{\vec{f}_\MV}{\vec{q}}\right]\right|_{eq}=\textcolor{red}{\pfrac{}{\vec{q}}\left.\left[\ddt\left(\pfrac{\MT_2}{\vecd{q}}\right)+\pfrac{\MT_2}{\vec{q}}\right]\right|_{eq}}+\textcolor{green}{\left.\pfrac{\vec{f}_\MV}{\vec{q}}\right|_{eq}}$}

\begin{itemize}
\item \textcolor{red}{All the terms will contain $\vecd{q},\vecdd{q}\ \Longrightarrow$ at the equilibrium this term vanishes.}
\item \textcolor{green}{$\left.\pfrac{\vec{f}_\MV}{\vec{q}}\right|_{eq}=\left.\frac{\partial^2\MV}{\partial\vec{q}\partial\vec{q}}\right|_{eq}$}
\end{itemize}

Thus the stiffness matrix can be described as:

\important{$\tvec{K}=\left.\frac{\partial^2\MV}{\partial\vec{q}\partial\vec{q}}\right|_{eq}$}

Further the stiffness matrix is symmetric:

\mportant{$K_{ij}=\frac{\partial^2\MV}{\partial\vec{q}_i\partial\vec{q}_j}=\frac{\partial^2\MV}{\partial\vec{q}_j\partial\vec{q}_i}=K_{ji}$}

For small motions, the potential energy is a quadratic form of the stiffness matrix:

\mportant{$\MV=\onha\vec{q}^T\tvec{K}\vec{q}$}

\textbf{Thus there is no need to know the EOM to linearize the system!}

\subsubsection{Linearized equations for rheonomic systems}

\important{$\tilde{\tvec{K}}=\left.\pfrac{(\vec{f}_\MT+\vec{f}_\MV)}{\vec{q}}=\pfrac{}{\vec{q}}\left[\ddt\pfrac{(T_0+T_1+T_2)}{\vecd{q}}-\pfrac{(T_0+T_1+T_2)}{\vec{q}}\right]\right|_{eq}+\left.\frac{\partial^2\MV}{\partial\vec{q}\partial\vec{q}}\right|_{eq}=\ldots+\textcolor{red}{\left.\frac{\partial^2\MV}{\partial\vec{q}\partial\vec{q}}\right|_{eq}}=\textcolor{red}{\tvec{K}}+\tvec{B}$}

\textcolor{red}{Same as for sceleronomic systems}

The extra contributions $\tvec{B}$ are typically centrifugal forces.

\vspace{3ex}

The mass matrix is left unchanged.

\important{$\tilde{\tvec{M}}=\pfrac{\vec{f}_\MT}{\vecdd{q}}=\ldots=\tvec{M}$}

The damping is found as:

\important{$\tilde{\tvec{D}}=\pfrac{\vec{f}_\MT}{\vecd{q}}=\pfrac{}{\vecd{q}}\left[\ddt\pfrac{(T_0+T_1+T_2)}{\vecd{q}}-\pfrac{(T_0+T_1+T_2)}{\vec{q}}\right]=\ldots=\left.\frac{\partial^2T_1}{\partial\vecd{q}\partial\vec{q}}-\frac{\partial^2T_1}{\partial\vec{q}\partial\vecd{q}}\right|_{eq}\neq\tvec{0}$}

The damping matrix is skew symmetric. (\textbf{Gyroscopic matrix}) It is usually called $\tvec{G}$.

\mportant{$D_{ij}=\frac{\partial^2T_1}{\partial\dot{q}_i\dot{q}_j}-\frac{\partial^2T_1}{\partial q_iq_j}=-D_{ji}$}

Note also that $\vecd{q}^T\tvec{G}\vecd{q}=0$, thus the power of gyroscopic forces is zero, they do not extract power from the system!

\important{$\tvec{M}\vecdd{q}+(\tvec{G}+\tvec{C})\vecd{q}+(\tvec{K}+\tvec{B})\vec{q}=\vec{0}$}

where $\tvec{C}$ accounts for physical damping.

\section{Vibrations}

\subsection{Stability of conservative systems}

Consider the case when $\MT = T_2(\vecd{q},\vec{q})$ and $\vec{Q}^{nc}=\vec{0}$.

\begin{align*}
\MT&=T_2(\vec{q},\vecd{q})\Longrightarrow \vecd{q}^t\pfrac{\MT}{\vecd{q}}=2T\\
&\text{Differentiating w.r.t. time}\\
2\pfrac{\MT}{t}&=\vecdd{q}^T\pfrac{\MT}{\vecd{q}}+\vecd{q}^t\ddt\left(\pfrac{\MT}{\vecd{q}}\right)\\
&\text{It also holds that:}\\
\pfrac{\MT}{t}&=\vecdd{q}^T\pfrac{\MT}{\vecd{q}}+\vecd{q}^T\pfrac{\MT}{\vec{q}}\\
&\text{Subtraction yields:}\\
\pfrac{\MT}{t}&=\vecd{q}^T\underbrace{\left[\ddt\left(\pfrac{\MT}{\vecd{q}}\right)-\pfrac{\MT}{\vec{q}}\right]}_{\text{Inertial terms of LE}}=\vecd{q}^T\vec{Q}=-\vecd{q}^T\pfrac{\MV}{\vec{q}}=-\frac{d\MV}{dt}\\
&\text{Therefore}\\
\frac{d\MT}{dt}&=-\frac{d\MV}{dt}\Longrightarrow\ddt(T+V)=0\Longrightarrow \MT+\MV=\ME
\end{align*}

Shift coordinates for simplicity: $\vec{q}_{eq}=\vec{0}$ and $\MV(\vec{q}_{eq})=0$ then:

\vspace{3ex}

$\vec{q}_{eq}$ is stable if there exists an energy bound $\ME^+$ such that for $\ME<\ME^+$ given to the system $\MT<\ME\ \forall t$ and $\MT=\ME$ at equilibrium.

\myspic{0.45}{Pictures/VibrationsEquilibrium}

Around equilibrium, for stability it must be $\MV>0$, then $\vec{q}_{eq}$ is a local minimum.

\mportant{$\MV=\onha\vec{q}^T\tvec{K}\vec{q}>0\ \forall\vec{q}$}

For a conservative, stable sceleronomic system the mass and the stiffness matrices are symmetric and positive definite.

\myspic{0.45}{Pictures/VibrationsEquilibrium2}

$\tvec{K}$ symmetric and positive definite: \textbf{All eigenvalues are real positive.}

\subsection{Motion around a stable equilibrium}

\mportant{$\tvec{M}\vecdd{q}+\tvec{K}\vec{q}=\vec{0}$}

The most general solution form is given by:

\mportant{$\vec{q}=\vec{X}e^{\lambda t},\ \vec{X}\in\mathbb{C}^n,\ \lambda\in\mathbb{C}$}

By substitution in the linearized equation:

\mportant{$(\lambda^2\tvec{M}+\tvec{K})\vec{X}=\vec{0}$}

Consider then the eigenvector $\vec{X}$ and its complex conjugate $\vec{X}^\ast$:

\mportant{$\vec{X}=\vec{a}+i\vec{b}\qquad\vec{X}^\ast=\vec{a}-i\vec{b}$}

By pre-multiplying the eigenvalue problem by $\vec{X}^\ast$ the eigenvalues are found as:

\mportant{$\lambda^2=-\frac{(\vec{a}-i\vec{b})^T\tvec{K}(\vec{a}+i\vec{b})}{(\vec{a}-i\vec{b})^T\tvec{M}(\vec{a}+i\vec{b})}$}

Recall that the mass and stiffness matrix are symmetric and positive definite:

\mportant{$\lambda^2=-\frac{\vec{a}^T\tvec{K}\,\vec{a}+\vec{b}^T\tvec{K}\,\vec{b}}{\vec{a}^T\tvec{M}\,\vec{a}+\vec{b}^T\tvec{M}\,\vec{b}}<0$}

Thus:

\important{$\lambda^2<0\Longrightarrow \lambda = i\omega$}

Thus the eigenvector $\vec{X}$ must be real as well and the eigenvalue problem can be rewritten as:

\mportant{$(\tvec{K}-\omega^2\tvec{M})\vec{X}=\vec{0}$}

The solutions then are pairs of eigenmodes and eigenfrequencies:

\mportant{$(\tvec{K}-\omega_p^2\tvec{M})\vec{X}_p=\vec{0},\ p=1,\ldots,n$}

$\vec{X}_p$ are the eigenmodes of the system.

The associated motion is purely oscillatory:

\important{$\vec{q}(t)=\vec{X}_pe^{i\omega_p t}=\vec{X}_p(\cos\omega_p t+i\sin\omega_p t)$}


\subsection{Mode orthogonality}

Consider two pairs of eigensolutions ($\vec{X}_k,\omega_k$) and $(\vec{X}_r,\omega_r)$ with $\omega_r\neq\omega_k$.

\mportant{$\vec{X}_k^T\tvec{K}\,\vec{X}_r=\omega_r^2\vec{X}_k^T\tvec{M}\,\vec{X}_r\qquad[\vec{X}_k^T\tvec{K}\,\vec{X}_r]^T=\vec{X}_r^T\tvec{K}\,\vec{X}_k=\vec{X}_r^T\tvec{K}\,\vec{X}_k$}

and for the second eigensolution:

\mportant{$\vec{X}_r^T\tvec{K}\,\vec{X}_k=\omega_k^2\vec{X}_r^T\tvec{M}\,\vec{X}_k$}

Subtraction and $\tvec{K}=\tvec{K}^T,\ \tvec{M}=\tvec{M}^T$ yield:

\mportant{$(\omega_k^2-\omega_r^2)\vec{X}_r\tvec{M}\,\vec{X}_k=\vec{0}$}

If $\omega_k\neq\omega_r$:

\important{$\vec{X}_r^T\tvec{M}\,\vec{X}_k=0\qquad\vec{X}_r^T\tvec{K}\,\vec{X}_k=0\quad r\neq k$}

Thus eigenmodes are orthogonal with respect to the mass and the stiffness matrix. Thus the modes are not zero when multiplied with the scalar product but actually are orthogonal based on this newly defined mass/stiffness-scalar product!

\importname{Modal mass}{$\vec{X}_k^T\tvec{M}\,\vec{X}_k=\mu_k$}

\importname{Modal stiffness}{$\vec{X}_k^T\tvec{K}\,\vec{X}_k=\gamma_k$}

The modes are only defined up to a constant. The ratio of modal mass and modal stiffness however is well defined!

\importname{Rayleigh quotient}{$\frac{\gamma_k}{\mu_k}=\frac{\vec{X}_k^T\tvec{K}\,\vec{X}_k}{\vec{X}_k^T\tvec{M}\,\vec{X}_k}=\omega_k^2$}

Note that the Rayleigh quotient is \emph{stationary} with respect to the vector. In other words, if the error in the estimated eigenvector is of order $\epsilon$, the error in the corresponding eigenfrequency is of order $\epsilon^2$:

\mportant{$R(\vec{X}_r)=\frac{\gamma_r+\mathcal{O}(\epsilon^2)}{\mu_r+\mathcal{O}(\epsilon^2)}\approx\frac{\gamma_r}{\mu_r}+\mathcal{O}(\epsilon^2)=\omega_r^2+\mathcal{O}(\epsilon^2)$}

When using a unit modal mass: $\mu_k = 1$

\begin{align*}
\vec{X}_k^T\tvec{K}\,\vec{X}_r &= \delta_{kr}\\
\vec{X}_k^T\tvec{M}\,\vec{X}_r&=\omega_k^2\delta_{kr}
\end{align*}

Note that, if $\omega_k = 0\Longrightarrow \tvec{K}\vec{X}_k=\vec{0}\quad (\tvec{K}-\omega^2\tvec{M})\vec{X}=\vec{0}$, the corresponding $\vec{X}_k$ is called rigid body mode.

The motion associated to one mode is given by:

\mportant{$\vec{q}=\vec{X}_j e^{i\omega_j t}\Longrightarrow\vecdd{q}=-\omega_j^2\vec{X}_j e^{i\omega_j t}$}

Then the orthogonality condition can be read as the work done by the inertial/elastic forces to one mode onto displacements of another mode:

\important{$\textcolor{green}{\vec{X}_j^T}\textcolor{red}{\tvec{K}\,\vec{X}_i}=0\qquad\textcolor{green}{\vec{X}_j^T}\textcolor{blue}{\tvec{M}\omega_i^2\vec{X}_i}=\omega_i^2\vec{X}_j^T\tvec{M}\,\vec{X}_i=0$}

\begin{itemize}
\item \textcolor{green}{Displacement affine to mode $\vec{X}_j$}
\item \textcolor{red}{Elastic force produced by mode $\vec{X}_i$}
\item \textcolor{blue}{Inertia forces produced by mode $\vec{X}_i$}
\end{itemize}

\subsection{Modal superposition}

\mportname{Forced linearized problem}{$\tvec{M}\,\vecdd{q}+\tvec{K}\,\vec{q}=\vec{p}(t)$}

The general solution can be expressed as a superposition of the modal shapes:

\mportant{$\vec{q}(t)=\sum\limits_{i=1}^n\underbrace {\eta_i(t)}_\text{Modal coordinate}\underbrace{\vec{X}_i}_{\text{Mode}}$}

Substitution and projection onto a generic mode $\vec{X}_j$ yields:

\mportant{$\vec{X}_j^T\left[\tvec{M}\sum\limits_{i=1}^n\ddot{\eta}_i(t)\vec{X}_i+\tvec{K}\sum\limits_{i=1}^n\eta_i(t)\vec{X}_i\right]=\vec{X}_j^T\vec{p}$}

Because of orthogonality the projection decouples the equations:

\mportant{$\vec{X}_j^T\tvec{M}\,\vec{X}_j\ddot{\eta}_j(t)+\vec{X}_j^T\tvec{K}\,\vec{X}_j\eta_j(t)=\tvec{X}_j^T\vec{p}\quad j=1,\ldots n$}

More compactly:

\mportant{$\ddot{\eta}_j+\omega_j^2\eta_j=\Phi_j\qquad\Phi_j=\frac{\vec{X}_j^T\,\vec{p}}{\mu_j}\quad j=1,\ldots,n$}

where $\Phi_j$ is the modal participation factor, ie.e. ``how much laod is contained in mode $\vec{X}_j$''.

In matrix form:

\mportant{$\tvec{I}\,\vecdd{\eta}+\tvec{\Omega}^2\vec{\eta}=\vec{\Phi}$}

\mportant{$\tvec{I}=\begin{bmatrix}
1&&\\&\ddots&\\&&1
\end{bmatrix}\quad\tvec{\Omega}=\begin{bmatrix}
\omega_1^2&&\\&\ddots&\\&&\omega_n^2
\end{bmatrix}\quad\vec{\eta}=\begin{bmatrix}
\eta_1\\\vdots\\\eta_n
\end{bmatrix}\quad\vec{\Phi}=\begin{bmatrix}
\Phi_1\\\vdots\\\Phi_n
\end{bmatrix}$}

To express the initial conditions in modal shapes:

\mportant{$\vec{q}(0)=\sum\limits_{i=1}^n\vec{X}_i\eta_i(0)\Longrightarrow\vec{X}_j^T\tvec{M}\sum\limits_{i=1}^n\vec{X}_i\eta_i(0)\textcolor{red}{\Longrightarrow}\eta_j(0)=\frac{\vec{X}_j^T\tvec{M}\,\vec{q}_0}{\mu_j}$}

\textcolor{red}{$\qquad\qquad\qquad$ Orthogonality}

\mportant{$\vecd{q}(0)=\sum\limits_{i=1}^n\vec{X}_i\dot{\eta}_i(0)\Longrightarrow\vec{X}_j^T\tvec{M}\vecd{q}_0=\vec{X}_j^T\tvec{M}\sum\limits_{i=1}^n\vec{X}_i\dot{\eta}_i(0)\textcolor{red}{\Longrightarrow}\dot{\eta}_j(0)=\frac{\vec{X}_j^T\tvec{M}\,\vecd{q}_0}{\mu_j}$}

If viscous damping is present: $\vecd{q}^T\tvec{C}\,\vecd{q}>0$

\mportant{$\tvec{M}\,\vecdd{q}+\tvec{C}\,\vecd{q}+\tvec{K}\vec{q}=\vec{p}$}

Decoupling cannot be achieved by projecting on modes, as

\mportant{$\vec{X}_i^T\tvec{C}\,\vec{X}_j\neq 0\qquad i\neq j$}

The modal equations would look like:

\mportant{$\ddot{\eta}_j+\sum\limits_{j=1}^n\theta_{ij}\dot{\eta}_j+\omega_i^2\eta_i=\Phi_i\qquad\theta_{ij}=\vec{X}_i^T\tvec{C}\,\vec{X}_j$}

Two remedies are possible:

\begin{enumerate}
\item Ignore the off-diagonal terms. When is this possible?
\item Rayleigh/proportional damping: $\tvec{C}=\alpha\tvec{M}+\beta\tvec{K}$
\end{enumerate}

\subsubsection{Including off-diagonal Terms}

Consider the damped linearized equation

\mportant{$\tvec{M}\vecdd{q}+\tvec{C}\vecd{q}+\tvec{K}\vec{q}=\vec{0}$}

Ansatz: $\vec{q}=\vec{Z}e^{\lambda t}$ which leads to the complex (difficult) eigenvalue problem:

\mportant{$(\lambda^2\tvec{M}+\lambda\tvec{C}+\tvec{K})\vec{Z}=\vec{0}$}

If damping is small: $||\tvec{C}\vecd{q}||<<||\tvec{K}\vec{q}||,||\tvec{M}\vecdd{q}||$ we can assume that:

\begin{flalign*}
\lambda_k&=\textcolor{green}{i\omega_k}+\textcolor{blue}{\Delta\lambda}\\
\vec{Z}_k&=\underbrace{\textcolor{green}{\vec{X}_k}}_\text{For the undamped system}+\underbrace{\textcolor{blue}{\Delta\vec{Z}}}_\text{Small correction}
\end{flalign*}

Substitution yields

\mportant{$\left[(-\omega_k^2+2i\omega_k\Delta\lambda+(\Delta\lambda)^2\tvec{M}+(i\omega_k+\Delta\lambda)\tvec{C}+\tvec{K}\right](\vec{X}_k+\Delta\vec{Z})=\vec{0}$}

Neglecting second order terms and using $(\tvec{K}-\omega_k^2\tvec{M})\vec{X}_k=\vec{0}$:

\mportant{$(\tvec{K}-\omega_k^2\tvec{M})\Delta\vec{Z}+i\omega_k(\tvec{C}+2\Delta\lambda\tvec{M})\vec{X}_k\simeq\vec{0}$}

Now projection onto $\vec{X}_K$ (same mode) yields:

\mportant{$\vec{X}_k^T(\tvec{C}+2\Delta\lambda\tvec{M})\vec{X}_k\simeq\vec{0}$}

which yields:

\important{$\Delta\lambda=-\frac{\theta_{kk}}{2\mu_k}$}

\mportant{$\lambda_k=i\omega_k-\frac{\theta_{kk}}{2\mu k}\qquad \theta_{kk}=\vec{X}_k^T\tvec{C}\,\vec{X}_k$}

\begin{itemize}
\item The correction to the eigenfrequency is real and negative. (Because $\tvec{C}$ is pos. def.)
\item First order analysis includes only diagonal terms of the modal damping matrix
\end{itemize}

For the correction of the mode:

\mportant{$\Delta\vec{Z}=\sum\limits_{s=1,\ s\neq k}^n\alpha_s\vec{X}_s\quad\alpha_s\in\mathbb{C}$}

where $s\neq k$ since correction of a mode with itself is meaningless, it would only stretch it. The direction of the mode is meaningful, the length is not.

Projection onto a generic mode $\vec{X}_j$:

\mportant{$\vec{X}_j^T\left[(\tvec{K}-\omega_k^2\tvec{M})\sum\limits_{s=1\ s\neq k}^n\alpha_s\vec{X}_s+i\omega_k(\tvec{C}+2\Delta\lambda\tvec{M})\vec{X}_k\right]\simeq 0$}

After accounting for all orthogonalities we get:

\mportant{$\alpha_s=\frac{i\omega_k\theta_{ks}}{\mu_s(\omega_k^2-\omega_s^2)}$}

This correction now includes off-diagonal terms! Also, if the eigenfrequencies $\omega_k$ and $\omega_s$ are well separated the correction becomes negligible.

If $\Delta Z$ is imaginary, the motion assigned to mode $k$ is:

\mportant{$\vec{q}_k=\vec{Z}_ke^{\lambda_k t}=(\vec{X}_k+i\vec{Q}_k)e^{\Delta\lambda+i\omega_k)t}\qquad \vec{Q}_k=\sum\limits_{p=1\ p\neq k}^n\frac{\omega_k\theta_{kp}}{\omega_k^2-\omega_p^2}\vec{X}_p$}

The motion is not synchronous anymore:

\begin{align*}
\vec{q}_k&=e^{\Delta\lambda t}e^{i\omega_kt}(\vec{X}_k-i\vec{Q}_k)=e^{\Delta\lambda t}(\cos\omega_k t+i\sin\omega_k t)(\vec{X}_k+i\vec{Q}_k)\\
&=e^{\Delta\lambda t}[\vec{X}_k\cos\omega_k t-\vec{Q}_k\sin\omega_kt+i\vec{X}_k\sin\omega_kt+i\vec{Q}_k\cos\omega_k t]
\end{align*}

We are only interested in the real part of the response:

\important{$\vec{q}_k=e^{\Delta\lambda t}[\vec{X}_k\cos\omega_kt-\vec{Q}_k\sin\omega_k t]$}

Thus $\vec{Q}_k$ and $\vec{X}_k$ are $\pi/2$ out of phase!

The modal equations can still be decoupled, if $||c||$ is small and the frequencies are well separated:

\mportant{$\ddot{\eta}_k+\frac{\theta_{kk}}{\mu_k}\dot{\eta}_k+\omega_k^2\eta_k=\Phi_k$}

which is analog to a single modal spring mass damper system with a spring of stiffness $\omega_k^2$, a damper of $\frac{\theta_{kk}}{\mu_k}$ and a mass of $\mu_k$.

\subsubsection{Integration of the modal equations}

For integration with \verb+ode45+ the equation needs to be transformed to a system of first order equations:

\begin{align*}
A\vec{z}+\vec{\Phi}_k&=\vecd{z}\\
\vec{z}&=\begin{bmatrix}
\dot{\eta}_k\\\eta_k
\end{bmatrix}\\
\begin{bmatrix}
\ddot{\eta}_k\\\dot{\eta}_k
\end{bmatrix}
&=\begin{bmatrix}
-\theta_kk/\mu_k&-\omega_k^2\\1&0
\end{bmatrix}\begin{bmatrix}
\dot{\eta}_k\\\eta_k
\end{bmatrix}+\begin{bmatrix}
\Phi_k\\0
\end{bmatrix}
\end{align*}

\subsubsection{Damping Ratio}

Or posing $\Theta_{kk}=2\epsilon_k\omega_k\mu_k$:

\mportant{$\ddot{\eta}_k+2\epsilon_k\omega_k\dot{\eta}_k+\omega_k^2\eta_k=\Phi_k\ \forall k=1,\ldots,n$}

Where typically $\epsilon_k$ are determined from modal testing.

\subsubsection{Rayleigh/Proportional Damping}

In practice, a damping model is often not available, or too complex to be derived. If one can assume small damping, it is then convenient to express the damping matrix as

\mportant{$\tvec{C}=a\tvec{K}+b\tvec{M}$}

This automatically guarantees orthogonality of the modes w.r.t. the damping matrix, and therefore decoupled modal equations. The corresponding modal damping coefficient is

\mportant{$\theta_k=a\gamma_k+b\mu_k$}

The damping ratio is then easily computed as:

\mportant{$\epsilon_k=\onha\left(a\omega_k+\frac{b}{\omega_k}\right)$}

Those can match up to two modes exactly, the rest will be estimated too highly or too lowly.

\subsubsection{Systems with arbitrary damping}

In a general case the state space description is necessary:

\mportant{$\begin{bmatrix}
\tvec{C}&\tvec{M}\\\tvec{M}&\tvec{0}
\end{bmatrix}
\ddt\begin{bmatrix}
\vec{q}\\\vecd{q}
\end{bmatrix}
+\begin{bmatrix}
\tvec{K}&\tvec{0}\\\tvec{0}&-\tvec{M}
\end{bmatrix}
\begin{bmatrix}
\vec{q}\\\vecd{q}
\end{bmatrix}
=\begin{bmatrix}
\vec{p}\\\vec{0}
\end{bmatrix}
$}

where

\mportant{$\tvec{A}=\begin{bmatrix}
\tvec{K}&\tvec{0}\\\tvec{0}&-\tvec{M}
\end{bmatrix}\quad\tvec{B}=\begin{bmatrix}
\tvec{C}&\tvec{M}\\\tvec{M}&\tvec{0}
\end{bmatrix}\quad \vec{r}=\begin{bmatrix}
\vec{q}\\\vecd{q}
\end{bmatrix};\quad\vec{s}(t)=\begin{bmatrix}
\vec{p}(t)\\\vec{0}
\end{bmatrix}$}

The system can be compactly written as:

\mportant{$\tvec{A}\vec{r}+\tvec{B}\vecd{r}=\vec{s}$}

The autonomous system (no load) can then be written as:

\mportant{$\tvec{A}\vec{r}+\tvec{B}\vecd{r}=\vec{0}$}

and it admits the general solution:

\mportant{$\vec{r}=\vec{y}e^{\lambda t}$}

Which leads to the eigenvalue problem (linear in $\lambda$):

\mportant{$\tvec{A}\vec{r}+\lambda\tvec{B}\vec{r}=\vec{0}$}

The solutions are pairs of complex eigenvalues and eigenvectors:

\mportant{$\lambda_k,\vec{y}_k,\ \forall k=1,\ldots,2n$}

Orthogonality of course still holds:

\mportant{$\vec{y}_j^T\tvec{A}\vec{y}_k=\tilde{\lambda}_k\delta_{jk}\quad\vec{y}_j^T\tvec{B}\vec{y}_k=\tilde{\mu}_k\delta{jk}$}

Note that, since the matrices $\tvec{A}$ and $\tvec{B}$ are real, the complex conjugate of the eigensolution is also an eigensolution

\begin{align*}
\lambda_k&=\sigma+i\nu\rightarrow\bar{\lambda}_k=\sigma-i\nu\\
\vec{y}_k&=\vec{u}+i\vec{v}\rightarrow\bar{\vec{y}}_k=\vec{u}-i\vec{v}
\end{align*}

The eigenvector can be also written as:

\mportant{$\vec{y}_k=\begin{bmatrix}
\vec{z}_k\\\lambda_k\vec{z}_k
\end{bmatrix}$}

The second order form of the eigenvalue problem is then:

\mportant{$\left(\lambda_k^2\tvec{M}+\lambda_k\tvec{C}+\tvec{K}\right)\vec{z}_k=\vec{0}$}

Premultiplying by the complex conjugate eigenvector

\mportant{$\bar{\vec{z}}_k^T(\lambda_k^2\tvec{M}+\lambda_k\tvec{C}+\tvec{K})\vec{z}_k=0\rightarrow\lambda_k^2m_k+\lambda_kc_k+k_k=0$}

where

\mportant{$k_k=\vec{\bar{z}}_k^T\tvec{K}\vec{z}_k\quad c_k=\vec{\bar{z}}_k^T\tvec{C}\vec{z}_k\quad m_k=\vec{\bar{z}}_k^T\tvec{M}\vec{z}_k$}

The general solution is:

\mportant{$\lambda_k=\frac{-c_k\pm\sqrt{c_k^2-4k_km_k}}{2m_k}$}

Or compactly

\mportant{$\lambda_k=-\alpha_k+i\omega_k$}

where

\mportant{$\alpha_k=\frac{c_k}{2m_k}\qquad\omega_k=\sqrt{\frac{k_k}{m_k}-\underbrace{\left(\frac{c_k}{2m_k}\right)^2}_\text{2nd order correction for small damping ass.}}$}

If $c_k<2\sqrt{k_km_k}$ the response is underdamped.

\subsection{Response to harmonic excitation}

\mportant{$\tvec{M}\vecdd{q}+\tvec{C}\vecd{q}+\tvec{K}\vec{q}=\vec{p}e^{i\Omega t}$}

where the damping $\tvec{C}\vecd{q}$ is small.

After a certain time the transient response has died out and the solution is dominated by the particular solution:

\mportant{$\vec{q}=\vec{Z}e^{i\Omega t}$}

Substitute that in the original equation:

\mportant{$[-\Omega^2\tvec{M}+i\Omega\tvec{c}+\tvec{K}]\vec{Z}=\vec{p}$}

Then the solution can be developed as a combination of undamped modes:

\mportant{$\vec{Z}=\sum\limits_{j=1}^n\alpha_j\vec{X}_j$}

Substitution yields:

\mportant{$[-\Omega^2\tvec{M}+i\Omega\tvec{C}+\tvec{K}]\sum\limits_{j=1}^n\alpha_j\vec{X}_j=\vec{p}$}

Pre-multiplication with $\vec{X}_k^T$ yields:

\mportant{$\vec{X}_k^T[-\Omega^2\tvec{M}+i\Omega\tvec{C}+\tvec{C}]\sum\limits_{j=1}^n\alpha_k\vec{X}_j=\vec{X}_k^T\vec{p}$}

Ans assuming small damping off-diagonal terms are neglected:

\mportant{$\alpha_k=\frac{\vec{X}_k^T\vec{p}}{(\omega_k-\Omega^2+2i\epsilon_k\Omega\omega_k)\mu_k}$}

Then the response is given by:

\important{$\vec{q}=\sum\limits_{k=1}^n\frac{\vec{X}_k^T\vec{p}}{(\omega_k^2-\Omega^2+2i\epsilon_k\Omega\omega_k)\mu_k}\vec{X}_k$}

Thus the response is the summation of all responses in modal coordinates transformed back by multiplication with the modal shapes.

\vspace{3ex}

\begin{enumerate}
\item For $\Omega=0\quad\alpha_k=\frac{\vec{X}_k^T\vec{p}}{\omega_k^2\mu_k}\Rightarrow$ Static response $(F/k)\quad \frac{\vec{X}_k^T\vec{p}}{\frac{\gamma_k}{\mu_k}\mu_k}$

\item For $\Omega=\omega_k\quad\alpha_k=\frac{\vec{X}_k^T\vec{p}}{2i\epsilon_k\Omega^2\mu_k}=-i\frac{\vec{X}_k^T\vec{p}}{2\epsilon_k\Omega^2\mu_k}$ Response is $\SI{90}{\degree}$ out of phase
\item For $\Omega>>\omega_k\quad\alpha_k\simeq-\frac{\vec{X}_k^T\vec{p}}{\Omega^2\mu_k}$ Response is $\SI{180}{\degree}$ out of phase
\end{enumerate}

\subsection{Anti-Resonance}

Assume the undamped case:

\mportant{$\vec{q}=\sum\limits_{k=1}^n\frac{\vec{X}_k^T\vec{p}}{(\omega_k^2-\Omega^2)\mu_k}\vec{X}_k$}

When a load is applied to the $j_{th}$ dof the ratio between the excitation and the same dof $a_{jj}$ is:

\mportant{$a_{jj}=\frac{q_j}{p_j}=\sum\limits_{k=1}^n\frac{X_{j(k)}^2}{(\omega_{(k)}^2-\Omega^2)\mu_{(k)}}$}

Its derivative w.r.t. the forcing frequency is always positive:

\mportant{$\pfrac{a_{jj}}{\Omega^2}=\sum\limits_{k=1}^n\frac{X_{j(k)}^2}{(\omega_{(k)}^2-\Omega^2)^2\mu_{(k)}}>0,\ \forall\Omega$}

At anti-resonance, the forcing applied to the degree of freedom of interest is in fact providing the necessary reaction force to keep this degree of freedom fixed.

\begin{itemize}
\item The rest of the system is moving.
\item The anti-resonance frequencies depend on the specific degree of freedom under consideration and the load.
\end{itemize}

\subsection{Response in the Time Domain}

Undamped case in modal equations:

\mportant{$\ddot{\eta}_k+\omega_k^2\eta_k=\Phi_k\quad k=1,\ldots,n\qquad \Phi_k=\frac{\vec{X}_k^T\vec{p}}{\mu_k}$}

Where the initial conditions are given by:

\mportant{$\dot{\eta}_k(0)=\frac{\vec{X}_k^T\tvec{M}\,\vecd{q}_0}{\mu_k}\qquad\eta_k(0)=\frac{\vec{X}_k^T\tvec{M}\,\vec{q}_0}{\mu_k}$}

To find the response the \textbf{Laplace transform} is used:

\important{$\ML(\eta(t))=\bar{\eta}_k(s)=\int_0^\infty e^{-st}\eta_k(t)dt\qquad \ML(\Phi(t))=\bar{\Phi}_k(s)=\int_0^\infty e^{-st}\Phi_k(t)dt$}

Application of the Laplace transform to the modal equation yields:

\mportant{$s^2\bar{\eta}_k-s\eta_k(0)-\dot{\eta}_k(0)+\omega_k^2\bar{\eta}_k=\bar{\Phi}_k$}

Which gives:

\mportant{$\bar{\eta}_k=\frac{s\eta_k(0)+\dot{\eta}_k(0)+\bar{\Phi}_k}{s^2+\omega_k^2}$}

The response will be given by the \textbf{inverse Laplace transform}:

\mportant{$\eta_k(t)=\ML^{-1}\left(\frac{s\eta_k(0)+\dot{\eta}_k(0)+\bar{\Phi}_k}{s^2+\omega_k^2}\right)$}

It can be shown that:

\mportant{$h_k(t)=\ML^{-1}\left(\frac{1}{s^2+\omega_k^2}\right)=\frac{\sin\omega_k t}{\omega_k}\qquad\ML^{-1}\left(\frac{s}{s^2+\omega_k^2}\right)=\cos\omega_k t$}

And therefore the response in time is given as:

\mportant{$\eta_k(t)=\eta_k(0)\cos\omega_k t+\dot{\eta}_k(0)\frac{\sin\omega_k t}{\omega_k}+\ML^{-1}\left(\frac{\bar{\Phi}_k}{s^2+\omega_k^2}\right)$}

\subsubsection{Duhamels integral}

\mportant{$\ddot{\eta}_k(0)+\omega_k^2\eta_k(0)=\Phi(0)\Rightarrow d\dot{\eta}_k(0)=\Phi(0)d\tau$}

The change in velocity due to an applied load at $t=0$ is proportional to the impulse of the load at the same time instant. Now $h_k(t)$ is used to shift the impulse to a generic time $\tau$ such that the whole arbitrary load can be written as a sum of impulses whose influence can be written as the \textbf{Duhamel's integral}:

\mportant{$d\eta_k(t)=\Phi_k(\tau)d\tau h_k(t-\tau)$}

\important{$\eta_k(t)=\int_0^t\Phi_k(\tau)h_k(t-\tau)d\tau$}

Thus the total time response is finally:

\important{$\eta_k(t)=\underbrace{\eta_k(0)\cos\omega_k t}_{\text{initial offset}}+\underbrace{\dot{\eta}_k(0)\frac{\sin\omega_kt}{\omega_k}}_{\text{initial velocity}}+\underbrace{\int_0^t\Phi_k(\tau)h_k(t-\tau)d\tau}_\text{forcing}$}

\subsection{Modal truncation}

Consider a forcing of the form: $\vec{p}=\vec{g}\phi(t)$

The physical response of the system is given by the superposition of the modal contribution as:

\mportant{$\vec{q}(t)=\sum\limits_{j=1}^n\frac{\vec{X}_j\vec{X}_j^T\vec{g}}{\mu_j}\frac{1}{\omega_j}\int_0^t\sin(\omega_j(t-\tau))\phi(\tau)d\tau$}

\definitiontable{
$\vec{X}_j$&The mode itself\\
$\vec{X}_j^T\vec{g}$&Projection onto the mode\\
$\frac{1}{\mu_j}$&Normalization by the normal mass
}

Together that makes the participation factor, that defines how important modes are for the modal truncation.

\begin{equation*}
\frac{\vec{X}_j\vec{g}}{\mu_j}
\end{equation*}

Now when using only $k<n$ modes for approximation the convergence of the summation is given by two factors:

\vspace{3ex}

\textbf{Spatial convergence:} The spatial distribution of the loads needs to be adequately represented by the retained modes:

\mportant{$\frac{\vec{X}_j\overbrace{\vec{X}_j^T\vec{g}}^\text{Load component along mode $\vec{X}_j$}}{\mu_j}$}

\textbf{Temporal convergence:} The response in time of each mode depend both on the eigenfrequency $\omega_j$ and the frequency content of the time history at which the load is applied:

\mportant{$\vec{r}_j(t)=\frac{1}{\omega_j}\int_0^t\sin(\omega_j(t-\tau))\underbrace{\phi(\tau)}_\text{load time history}d\tau$}

For instance for a step load: $\phi(t)=1,\ t>0$

\mportant{$\vec{r}_j(t)=\frac{1-\cos\omega_j t}{\omega_j^2}$}

The contribution of each mode decays with the square of the eigenfrequency. For the harmonic case $\phi(t)=\cos(\Omega t)$:

\mportant{$\vec{r}_j(t)=\frac{\omega_j\sin\Omega t-\Omega\sin\omega_j t}{\omega_j^2(\omega_j^2-\Omega^2)}$}

\subsection{Modal acceleration}

High frequency modes respond statically to the applied load. Thus only the first k modes are kept:

\mportant{$\xcancel{\ddot{\eta}_j}+\omega_j^2\eta_j=\frac{\vec{X}_j^T\vec{p}}{\mu_j}\rightarrow\omega_j^2\eta_j\approx\frac{\vec{X}_j^T\vec{p}}{\mu_j}\ j=k+1,\ldots,n$}

The response is then written as a superposition of the first k modes:

\mportant{$\vec{q}(t)=\sum\limits_{j=1}^k\vec{X}_j\eta_j(t)+\sum\limits_{j=k+1}^n\vec{X}_j\frac{\vec{X}_j^t\vec{p}(t)}{\omega_j^2\mu_j}$}

The higher frequency modes are typically not computed. The second term of the equation above is thus not available. However the static response of the whole system is given by:

\mportant{$\vec{q}_{stat}(t)=\tvec{K}^{-1}\vec{p}(t)=\sum\limits_{j=1}^n\vec{X}_j\frac{\vec{X}_j^T\vec{p}(t)}{\omega_j^2\mu_j}$}

And therefore the response can be written as:

\important{$\vec{q}(t)=\sum\limits_{j=1}^n\vec{X}_j\eta_j(t)+\tvec{K}^{-1}\vec{p}(t)-\sum\limits_{j=1}^k\vec{X}_j\frac{\vec{X}_j^T\vec{p}(t)}{\omega_j^2\mu_j}$}

\subsection{Response to base motion}

If a motion is prescribed to some of the available DOFs it makes sense to partition the system of differential equations as follows:

\mportant{$\begin{bmatrix}
M_{11}&M_{12}\\M_{21}&M_{22}
\end{bmatrix}\begin{bmatrix}
\ddot{q}_1\\\ddot{q}_2
\end{bmatrix}+\begin{bmatrix}
K_{11}&K_{12}\\K_{21}&K_{22}
\end{bmatrix}\begin{bmatrix}
q_1\\q_2
\end{bmatrix}=\begin{bmatrix}
0\\r_2
\end{bmatrix}$}

where $r_2$ are the reaction forces to impose the motion.

The first block can be solved as:

\mportant{$M_{11}\ddot{q}_1+K_{11}q_1=-M_{12}\ddot{q}_2-K_{12}q_{2}$}

and then the second for the reaction forces:

\mportant{$r_2=M_{21}\ddot{q}_1+M_{22}\ddot{q}_2+K_{21}q_1+K_{22}q_2$}

This procedure requires the knowledge of the prescribed motion and the acceleration.

A different strategy would be to decompose the motion of the unforced DOFs in

\begin{enumerate}
\item the static displacement due to the imposed motion $q_2$
\item its dynamics around the static deflection
\end{enumerate}

\mportant{$q_1=q_1^{st}+y_1$}

The static deflection can be found by neglecting accelerations in the first block equation:

\mportant{$q_1^{st}=-K_{11}^{-1}K_{12}q_2=\underbrace{S}_\text{Schur's complement}q_2$}

By introducing the transformation:

\mportant{$q(t)=\begin{bmatrix}
I&S\\0&I
\end{bmatrix}\begin{bmatrix}
y_1\\q_2
\end{bmatrix}\qquad\qquad \ddot{q}(t)=\begin{bmatrix}
I&S\\0&I
\end{bmatrix}\begin{bmatrix}
\ddot{y}_1\\\ddot{q}_2
\end{bmatrix}$}

One obtains

\mportant{$M_{11}\ddot{y}_1+K_{11}y_1=-[M_{11}S+M_{12}]\ddot{q}_2$}

This expression requires only the imposed acceleration and no the imposed displacements. this las euqation can be solved by modal superposition using few modes relative to the structure with fixed support:

\mportant{$M_{11}\ddot{y}_1+K_{11}y_1=0\rightarrow(-\omega_j^2M_{11}+K_{11})X_j=0$}

\subsection{Force appropriation testing}

To find modes, eigenfrequencies and damping ratios experimentally, a certain force $\vec{p}$  is applied to the system harmonically. For small damping:

\mportant{$[-\Omega^2\tvec{M}+i\Omega\tvec{C}+\tvec{K}]\vec{Z}=\vec{p}$}

By this we want to find a load distribution such that the $k^{th}$ undamped eigenmode $\vec{X}_k$ is excited $\Longrightarrow\ \Omega=\omega_k$ where $\vec{p}=\vec{p}_k$ is an excitation distribution that allows to excite $\vec{X}_k$.

\mportant{$[\tvec{K}-\omega_k^2\tvec{M}+i\omega_k\tvec{C}]\vec{X}_k=\vec{p}_k$}

As the result of the eigenvalue problem for the undamped system we know that ${[\tvec{K}-\omega_k^2\tvec{M}]\vec{X}_k=\vec{0}}$. Therefore:

\important{$\vec{p}_k=i\omega_k\tvec{C}\vec{X}_k$}

which means that we can force the system in such a way that the excitation force balances the damping force and the system oscillates as if undamped. Also since the system response is purely real and the excitation force is purely imaginary the excitation force is $\SI{90}{\degree}$ delayed. With time both forces will rotate in the complex plane but remain perpendicular to each other.

\myspic{0.6}{ComplexPlane}

\subsection{Non Appropriation Testing}

\begin{enumerate}
\item The structure is excited either by sine-sweep or by impact hammer at discrete points. Note that the impact hammer is supposed to implement a dirac-delta peak which by mathematical definition (using fft) is a uniform combination of all frequencies.
\item The response at discrete locations is measured.
\item The FFT of the excitation and the response gives the value of the frequency transfer function at the measured points.
\item One then tries to fit a FRF with the obtained measured data and determines

\mportant{$\vec{X}_k,\ \omega_k,\ \mu_k,\ \epsilon_k$}

\mportant{$\vec{q}=\sum\limits_{k=1}^n\frac{\vec{X}_k^T\vec{p}}{(\omega_k^2-\Omega^2+2i\epsilon_k\Omega\omega_k)\mu_k}\vec{X}_k$}

\begin{itemize}
\item This assumes linearity of the response.
\item The order of the model needs to be decided. Typically, the order is increased until the modes of interest have converged.
\item Requires much less equipment (hammer and a few accelerometers) but numerical and fitting techniques.
\end{itemize}
\end{enumerate}

\end{multicols*}

\end{document}